\chapter{Objectieve Beoordeling Muziekstuk}
\label{hoofdstuk:OBM}
Om melodische transformaties te kunnen beoordelen is er nood aan een framework dat kan voorspellen of een gegeven melodie al dan niet goed klinkt. In dit hoofdstuk worden twee zo een modellen besproken die een bepaalde `consonantiescore' (maat voor het goed klinken van een muziekstuk) gaan toekennen aan een muziekstuk. Allereerst wordt een aanpak besproken die gebruik maakt van een artifici\"eel neuraal netwerk \cite{book:ANN}. Deze methode is speciaal gemaakt voor het beoordelen van polyfone muziek, waarbij er meerdere noten tegelijk klinken. Polyfone muziek wordt in verdere hoofdstukken van deze masterproef niet meer besproken. Het eerste onderdeel van dit hoofdstuk zal bijgevolg dus weinig invloed hebben op het vervolg van de tekst. De methode is echter wel onderzocht geweest en leverde enkele interessante resultaten op waardoor de methode toch even in het kort vermeld wordt. Als tweede methode wordt een zogenaamd `RPK-Model' besproken dat gebaseerd is op een gelijknamig model uit het boek `Music and Probability' \cite{book:musicAndProbability} van David Temperley \cite{url:temperley}. Dit model werkt op muziek met een enkele melodielijn en zal ook het model zijn dat als verificator gebruikt zal worden bij de experimenten in het vervolg van deze masterproef.

\section{Neuraal Netwerk}
\label{OBM:NN}
In het boek `Musimathics' \cite{book:musimathics} wordt er een gedeelte gewijd aan het meten van de consonantie van akkoorden (samenklank van meerdere noten). Dit gebeurt kort samengevat door het trainen van een neuraal netwerk op zogenaamde tweeklanken (twee noten die tegelijkertijd gespeeld worden). En het neuraal netwerk maakt dan de veralgemening naar hogere orde samenklanken. Het neuraal netwerk dat gemaakt getest werd als verificator bestond uit drie lagen. een invoerlaag, een uitvoerlaag en dan nog een verborgen laag. De invoerlaag bestond uit 12 neurons die elk voor een van de 12 verschillende noten staan. Twee noten die een octaaf uit elkaar liggen (en muziektechnisch dezelfde naam krijgen) zullen dus ook op eenzelfde neuron afgebeeld worden. Deze 12 input neurons krijgen een `1' als input wanneer er een noot in het akkoord zit dat afgebeeld wordt op die neuron. In het andere geval krijgt deze een `0' als input. Vervolgens komen we op een hidden layer terecht die volledige geconnecteerd is en uit 6 neurons bestaat. Tot slot komen we nog bij de output layer uit die uit slechts 1 neuron bestaat. Deze neuron gaat een getal tussen 0 en 1 teruggeven, hoe dichter het getal bij 1 ligt hoe zekerder het neuraal netwerk is dat het ingegeven akkoord consonant is. Hoe dichte bij 0 hoe zekerder hij is dat het akkoord dissonant (tegengestelde van consonant) is.\\
Dit netwerk werd getraind op alle mogelijke tweeklanken gebruik makend van de \textit{backpropagation of error}-methode\cite{url:backpropagation}. Voor elke mogelijke tweeklank werd afhankelijk van de afstand (in halve tonen) tussen de twee noten bepaald of ze goed samen klinken of niet. Dit volgens de regels van de muziektheorie. Deze regels komen ook overeen met de verhoudingen van de frequenties van de twee invoernoten. Hoe kleiner de getallen in de vereenvoudigde breuk van de frequenties, hoe beter het akkoord klinkt. Na het trainen van het neuraal netwerk kan dit gebruikt worden om meerstemmige muziek te gaan beoordelen. Op elk tijdstip waarop er noten gespeeld worden kan men deze noten als input in het neuraal netwerk steken. De uitvoer van het netwerk geeft dan een maat voor de consonantie terug. Als we dit doen voor elk tijdstip in het muziekstuk waarop er noten gespeeld worden en we nemen dan het gemiddelde over al deze tijdstippen krijgen we een maat voor de `gemiddelde consonantie' van het gehele muziekstuk.\\
Nadat het neuraal netwerk getraind werd, bleek het neuraal netwerk goed te generaliseren naar akkoorden met meer dan 2 noten. Dit wil zeggen dat akkoorden van meer dan twee noten die muziektheoretisch `goed' zouden moeten klinken ook door het netwerk als dusdanig beoordeeld werden. Het netwerk kon dus dienen als een soort van alternatieve voorstelling van de regels van de muziektheorie wat de samenklank van akkoorden betreft. Het probleem lag er echter in dat de klassieke werken van Bach \cite{url:bach} en Mozart \cite{url:mozart} waarop getest werd zelf helemaal niet zo consonant waren als oorspronkelijk gedacht. En dit in die mate dat vaak tot een derde van de akkoorden in zo een stuk als dissonant (tegengestelde van consonant) bestempeld worden. Deze akkoorden blijken ook daadwerkelijk dissonant te zijn. Het is slechts door de context, de noten die net voor het akkoord gespeeld worden, dat deze akkoorden in die stukken toch niet als dissonant ervaren worden door de luisteraar. Het neuraal netwerk dat hier gebruikt werd houdt echter geen rekening met deze context. Dit leidt er wel toe dat dit neuraal netwerk niet gebruikt kan worden als verificator, aangezien zelfs muziekstukken van Bach en Mozart door de verificator niet als consonant herkend worden. Het is daarentegen wel nog maar eens een bevestiging van het genie van componisten als Bach en Mozart, de kunst ligt het niet in het volgen van de regels maar in het weten wanneer en hoe de regels gebroken mogen worden.\\
De broncode die geschreven werd voor het trainen van dit neuraal netwerk is beschikbaar in appendix \ref{Broncode:ANN}. Er zijn een aantal parameters die ingesteld kunnen worden in dit algoritme. Er is allereerst het aantal neuronen in de verborgen laag. Er kan ook een waarde voor de `learning rate' opgegeven worden en dan zijn er nog twee parameters die een bias kunnen defini\"eren.

\section{RPK-Model}
\label{OBM:RPK}

%%% Local Variables: 
%%% mode: latex
%%% TeX-master: "masterproef"
%%% End: 
