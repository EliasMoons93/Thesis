\chapter{Besluit}
\label{hoofdstuk:B}
Een hoofdstuk behandelt een samenhangend geheel dat min of meer op zichzelf
staat. Het is dan ook logisch dat het begint met een inleiding, namelijk
het gedeelte van de tekst dat je nu aan het lezen bent.

\section{Eerste onderwerp in dit hoofdstuk}
De inleidende informatie van dit onderwerp.

\subsection{Een item}
Een tekst staat nooit alleen. Dit wil zeggen dat er zeker ook referenties
nodig zijn. Dit kan zowel naar on-line documenten\cite{wiki} als naar
boeken\cite{pratchett06:_good_omens}.

\section{Figuren}
Figuren worden gebruikt om illustraties toe te voegen. Dit is dan ook de
manier om beeldmateriaal toe te voegen zoals getoond wordt in
figuur~\ref{fig:logo}.

\begin{figure}
  \centering
  \includegraphics{logokul}
  \caption{Het KU~Leuven logo.}
  \label{fig:logo}
\end{figure}

\section{Tabellen}
Tabellen kunnen gebruikt worden om informatie op een overzichtelijke te
groeperen. Een tabel is echter geen rekenblad! Vergelijk maar eens
tabel~\ref{tab:verkeerd} en tabel~\ref{tab:juist}. Welke tabel vind jij het
duidelijkst?

\begin{table}
  \centering
  \begin{tabular}{||l|lr||} \hline
    gnats     & gram      & \$13.65 \\ \cline{2-3}
              & each      & .01 \\ \hline
    gnu       & stuffed   & 92.50 \\ \cline{1-1} \cline{3-3}
    emu       &           & 33.33 \\ \hline
    armadillo & frozen    & 8.99 \\ \hline
  \end{tabular}
  \caption{Een tabel zoals het niet moet.}
  \label{tab:verkeerd}
\end{table}

\begin{table}
  \centering
  \begin{tabular}{@{}llr@{}} \toprule
    \multicolumn{2}{c}{Item} \\ \cmidrule(r){1-2}
    Animal    & Description & Price (\$)\\ \midrule
    Gnat      & per gram    & 13.65 \\
              & each        & 0.01 \\
    Gnu       & stuffed     & 92.50 \\
    Emu       & stuffed     & 33.33 \\
    Armadillo & frozen      & 8.99 \\ \bottomrule
  \end{tabular}
  \caption{Een tabel zoals het beter is.}
  \label{tab:juist}
\end{table}

\section{Besluit van dit hoofdstuk}
Als je in dit hoofdstuk tot belangrijke resultaten of besluiten gekomen
bent, dan is het ook logisch om het hoofdstuk af te ronden met een
overzicht ervan. Voor hoofdstukken zoals de inleiding en het
literatuuroverzicht is dit niet strikt nodig.

%%% Local Variables: 
%%% mode: latex
%%% TeX-master: "masterproef"
%%% End: 
