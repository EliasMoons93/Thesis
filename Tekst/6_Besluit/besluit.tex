\chapter{Besluit}
\label{hoofdstuk:B}

\section{Samenvatting}
In deze masterproef zijn allereerst twee methoden besproken voor de evaluatie van muziekstukken. Het gebruik van een neuraal netwerk voor polyfone muziekstukken had zijn tekortkomingen door het negeren van de context van een noot. Dit probleem werd opgelost door het RPK-model dat geschikt is voor monofone muziekstukken. Dit RPK-model kon daarna gebruikt worden voor de melodische evaluatie van verschillende transformaties.

Er werden twee melodische transformaties besproken. Een van deze transformaties diende vooral als \textit{baseline} en was zeer elementair en eenvoudig te implementeren. De andere transformatie dat zich baseert op de intervallen tussen opeenvolgende noten leverde betere resultaten (afwijking consonantiescore van de originele melodie ten opzichte van afwijking met een willekeurige transformatie is significant kleiner). De resultaten zijn, alhoewel ze duidelijk beter zijn dan in het willekeurige geval, wel nog steeds niet goed genoeg om te kunnen spreken van een ``goede'' transformatie. Dit komt omdat de bekomen consonantiescore van een stuk na deze transformatie gemiddeld gezien nog steeds significant minder goed is ten opzichte van die van het origineel.

Tot slot zijn er nog twee algoritmen ontworpen die zorgen voor het effici\"ent combineren van verschillende transformaties. Het eerste van deze algoritmes kan gegeven een aantal transformaties en een originele melodielijn, deze melodielijn transformeren naar een nieuwe melodielijn met een zo hoog mogelijke consonantiescore. Het tweede algoritme verwezenlijkt hetzelfde doel, maar heeft als extra voorwaarde dat een transformatie, wanneer deze wordt uitgevoerd, meteen voor een minimum aantal opeenvolgende noten uitgevoerd moet worden. Van deze algoritmen werd de afhankelijkheid van de verschillende parameters getest en in kaart gebracht. Van het tweede algoritme werd ook de berekende tijdscomplexiteit geverifi\"eerd. Deze algoritmen kunnen dus gebruikt worden om zo effici\"ent mogelijk (volgens het RPK-model) verschillende transformaties te combineren. Hierbij dient wel in het achterhoofd gehouden te worden dat het in deze algoritmen de bedoeling is de consonantiescore zo hoog mogelijk te krijgen. Dit leidt vaak niet tot interessante muziekstukken. Het juist kiezen van de parameters waarvan deze algoritmen afhankelijk zijn kan er wel voor zorgen dat de score gemiddeld gezien sneller of trager afwijkt van die van het origineel.

\section{Verder onderzoek}
In dit onderdeel worden nog enkele onderwerpen aangekaart die onderwerp zouden kunnen zijn voor verder onderzoek in dit onderzoeksdomein.

\subsection{Uitbreiding context RPK-model}
Het RPK-model brengt momenteel voor het berekenen van de probabiliteit van een muziekstuk voor elke noot, de voorafgaande noot in rekening. Het kan interessant zijn dit uit te breiden naar meerdere vorige noten aangezien deze, zij het in mindere mate, ook een invloed hebben op de waarschijnlijkheid van de huidige noot.\\
Momenteel wordt er ook enkel melodisch ge\"evalueerd, het ritme van het muziekstuk wordt namelijk genegeerd. Het zou ook interessant kunnen zijn, zelfs voor het louter evalueren van melodische transformaties, om ook het ritme in rekening te brengen. Men zou bijvoorbeeld kunnen verwachten dat hoe sneller de noten elkaar opvolgen hoe kleiner de gemiddelde sprongen in toonhoogte zullen zijn. Hierdoor zou de \textit{proximity}-component mede afhankelijk van de ritmische afstand kunnen gemaakt worden in plaat van enkel van de melodische afstand.

\subsection{Algoritme dat combineert met als doel het behouden van de consonantiescore}
De algoritmen die ontworpen zijn in dit onderzoek hebben als doel om de consonantiescore van een melodielijn zo hoog mogelijk te krijgen door het combineren van verschillende transformaties. Dit levert echter niet altijd interessante resultaten op in de praktijk. In het onderzoek is gesteld dat een transformatie als `goed' beschouwd wordt wanneer deze de consonantiescore zou weining mogelijk zou veranderen. In dit opzicht is het interessant om een algoritme te maken dat als doel zou hebben om een gegeven melodielijn te transformeren, gebruik makende van de beschikbare transformaties, naar een nieuwe melodielijn wiens consonantiescore die van het origineel zo dicht mogelijk benadert. Als extra voorwaarde zou dan kunnen opgelegd worden dat het nieuwe muziekstukje op melodisch vlak zo veel mogelijk zou moeten afwijken van het origineel zodat dit niet meer herkenbaar zou zijn na transformatie.

\subsection{Combinatie ritmische en melodische transformaties}
Deze masterproef beschrijft enkel melodische transformaties. Het kan echter interessant zijn om in een verder stadium van het onderzoek van muzikale transformaties te gaan kijken naar combinaties van melodische en ritmische transformaties. Zo zal een melodische transformatie bijvoorbeeld ook minder van het originele muziekstuk moeten aanpassen opdat het geheel niet aan het origineel zou doen denken. Dit omdat er ook nog een ritmische aanpassing zou zijn.

%%% Local Variables: 
%%% mode: latex
%%% TeX-master: "masterproef"
%%% End: 
