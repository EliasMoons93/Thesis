% 
% Sjabloon voor de master ingenieurswetenschappen: computerwetenschappen
%
\documentclass[master=cws]{kulemt}
%%%%%%%%%%%%%%  Wijzig niets boven deze regel  %%%%%%%%%%%%%
% 
% Vul de titel van jouw masterproef hieronder in tussen { en }.
\setup{title={Melodische transformatie en evaluatie van muziek},
%
% Vul hieronder namen in, steeds Voornaam Naam.
% Indien meerdere auteurs, assessoren, assistenten, scheidt hun namen
%   met \and .
  author={Elias Moons},
  promotor={Prof.\,dr.\,D. De Schreye},
  assessor={Prof. Onbekend},
  assistant={Ir.\ V. Nys}}
% 
\setup{filingcard,   % Deze regel niet wijzigen
%
% Vul de vertaalde titel van jouw masterproef hieronder in tussen { en }.
  translatedtitle={Melodical transformation en evaluation of music},
%
% UDC nummer is richtingafhankelijk. 
% Zie http://www.udcc.org/udcsummary/php/index.php
% voor het UDC nummer.
% Dit voorbeeld (519.6) verwijst naar 'Computational mathematics'
  udc=004.9,
%
% Hieronder, tussen { en } een korte samenvatting toevoegen.
% Lege regels en het commando \par zijn niet toegelaten.
% Wees voorzichtig met speciale TeX-tekens #$%&^_~{}\ !!
  shortabstract={%
    Deze masterproef beschrijft methodes om melodielijnen van een muziekstuk melodisch te transformeren to nieuwe melodielijnen. Verder wordt er ook aandacht besteed aan een referentiekader waarin deze transformaties ge\"evalueerd kunnen worden. Tot slot wordt er ook nog gekeken naar wanneer bepaalde transformaties nuttig kunnen zijn om de consonantie van een muziekstuk te verhogen. En hoe verschillende transformaties effici\"ent gecombineerd kunnen worden. Om dit te verwezenlijken is een algoritme beschreven dat gebaseerd is op de principes van \textit{dynamic programming}. Dit algoritme zal, gegeven een aantal mogelijke transformaties en een melodielijn, de best mogelijke getransformeerde melodielijn teruggeven volgens het gedefini\"eerde referentiemodel.
   }}
% Verwijder de "%" op de volgende lijn als je de kaft wil afdrukken
%\setup{coverpageonly}
% Verwijder de "%" op de volgende lijn als je enkel de eerste pagina's wil
% afdrukken en de rest bv. via Word aanmaken.
%\setup{frontpagesonly}

% Kies de fonts voor de gewone tekst, bv. Latin Modern
\setup{font=lm}

% Hier kun je dan nog andere pakketten laden of eigen definities voorzien

% Tenslotte wordt hyperref gebruikt voor pdf bestanden.
% Dit mag verwijderd worden voor de af te drukken versie.
\usepackage[pdfusetitle,colorlinks,plainpages=false]{hyperref}

%%%%%%%
% Om wat tekst te genereren wordt hier het lipsum pakket gebruikt.
% Bij een echte masterproef heb je dit natuurlijk nooit nodig!
\IfFileExists{lipsum.sty}%
 {\usepackage{lipsum}\setlipsumdefault{11-13}}%
 {\newcommand{\lipsum}[1][11-13]{\par Hier komt wat tekst: lipsum ##1.\par}}
%%%%%%%

%Eigen toegevoegde packages
\usepackage{pdfpages}
\usepackage{amsmath} %tijdssignatuur muziekstuk
\usepackage{xr} %cross referencing tussen verschillende tex files
\externaldocument{1_Muzikale_Achtergrond/muzikale_achtergrond}
\externaldocument{2_Objectieve_Beoordeling/objectieve_beoordeling}
\externaldocument{3_Melodische_Transformatie/melodische_transformatie}
\externaldocument{4_Efficient_Toepassen_Transformatie/efficient_toepassen_transformatie}
\externaldocument{5_Experimenten_Resultaten/experimenten_resultaten}
\externaldocument{6_Besluit/besluit}
\externaldocument{Appendix_0_Broncode/broncode}
\usepackage{listings}
\lstset{breaklines}
\usepackage{tikz}

\begin{document}

\begin{preface}
  TODO: iedereen bedanken...
\end{preface}

\tableofcontents*

\begin{abstract}
Deze masterproef beschrijft methodes om melodielijnen van een muziekstuk melodisch te transformeren to nieuwe melodielijnen. Verder wordt er ook aandacht besteed aan een referentiekader waarin deze transformaties ge\"evalueerd kunnen worden. Tot slot wordt er ook nog gekeken naar wanneer bepaalde transformaties nuttig kunnen zijn om de consonantie van een muziekstuk te verhogen. En hoe verschillende transformaties effici\"ent gecombineerd kunnen worden. Om dit te verwezenlijken is een algoritme beschreven dat gebaseerd is op de principes van \textit{dynamic programming}. Dit algoritme zal, gegeven een aantal mogelijke transformaties en een melodielijn, de best mogelijke getransformeerde melodielijn teruggeven volgens het gedefini\"eerde referentiemodel.
\end{abstract}

% Een lijst van figuren en tabellen is optioneel
%\listoffigures
%\listoftables
% Bij een beperkt aantal figuren en tabellen gebruik je liever het volgende:
\listoffiguresandtables
% De lijst van symbolen is eveneens optioneel.
% Deze lijst moet wel manueel aangemaakt worden, bv. als volgt:
%\chapter{Lijst van afkortingen en symbolen}
%\section*{Afkortingen}
%\begin{flushleft}
%  \renewcommand{\arraystretch}{1.1}
%  \begin{tabularx}{\textwidth}{@{}p{12mm}X@{}}
%    LoG   & Laplacian-of-Gaussian \\
%    MSE   & Mean Square error \\
%    PSNR  & Peak Signal-to-Noise ratio \\
%  \end{tabularx}
%\end{flushleft}
%\section*{Symbolen}
%\begin{flushleft}
%  \renewcommand{\arraystretch}{1.1}
%  \begin{tabularx}{\textwidth}{@{}p{12mm}X@{}}
%    42    & ``The Answer to the Ultimate Question of Life, the Universe,
%            and Everything'' volgens de \cite{h2g2} \\
%    $c$   & Lichtsnelheid \\
%    $E$   & Energie \\
%    $m$   & Massa \\
%    $\pi$ & Het getal pi \\
%  \end{tabularx}
%\end{flushleft}

% Nu begint de eigenlijke tekst
\mainmatter

\chapter{Inleiding}
\label{hoofdstuk:I}
%In dit hoofdstuk wordt het werk ingeleid. Het doel wordt gedefinieerd en er
%wordt uitgelegd wat de te volgen weg is (beter bekend als de rode draad).

%Als je niet goed weet wat een masterproef is, kan je altijd
%Wikipedia\cite{wiki} eens nakijken.

\section{Probleemstelling}
Een van de meest voorkomende problemen voor muzikanten is de zogenaamde \textit{writer's block}. Dit fenomeen waarbij je als muzikant merkt dat je de hele tijd op hetzelfde melodietje uitkomt en maar niet met iets nieuw kan komen kan zeer frustrerend zijn. Daarom zou het handig zijn als er een tool zou bestaan die je als het ware de inspiratie kan geven die je nodig hebt om deze \textit{writer's block} te doorbreken. Een mogelijke oplossing voor dit probleem wordt in deze thesis onderzocht. Hier gaat het dan om het transformeren van reeds gekende melodie\"en tot nieuwe melodie\"en.

\section{Vergelijking met voorgaand onderzoek}
Onderzoek naar het bekomen van nieuwe melodie\"en waarbij gebruik gemaakt wordt van een computer is niet nieuw \cite{thesis:thomas} \cite{book:musicAndProbability}. Al gedurende tientallen jaren is er veel werk geleverd op het vlak van muziekgeneratie, waarbij het de bedoeling is om een muziekstuk te genereren waarbij vertrokken wordt van een lege partituur \cite{book:musicGeeksNerds}. Dit wordt gedaan rekening houdend met de regels uit de muziektheorie en vaak probeert men er ook een vorm van repetitiviteit en herkenning in te brengen zoals vaak ook het geval is in hedendaagse muziek. 

Een groot probleem dat echter altijd terugkwam was dat het heel moeilijk was om een goede verhouding te vinden tussen twee heel belangrijke eigenschappen van de muziek, namelijk verwachting en verrassing van de luisteraar \cite{book:musimathics}. Een muziek luisteraar heeft namelijk een bepaald verwachtingspatroon voor het onmiddellijke vervolg van een melodie. Dit verwachtingspatroon strookt vaak met de regels van de muziektheorie. Wanneer deze regels dan weer te nauwgezet gevolgd worden, wordt het muziekstuk als saai ervaren, er is dus nood aan een zekere verrassing in de melodie. Te veel onverwachte wendingen worden dan weer als frustrerend gezien, kortom, het is van cruciaal belang om een goed evenwicht te vinden tussen deze twee waarden. Dit blijkt zeer moeilijk te verwezenlijken vanuit het standpunt van muziekgeneratie.\\

\section{Doelstelling}
In deze thesis wordt een onderzoek gevoerd naar de generatie van nieuwe melodie\"en vertrekkende van originele melodie\"en. Deze originele melodie\"en worden bij aanname verondersteld te voldoen aan de eisen van verrassing en verwachting. In deze thesis worden specifiek melodische transformaties onderzocht. Deze transformaties zullen gebaseerd zijn op wiskundige reeksen. Hierbij gaat men voor elke noot in de oorspronkelijke melodie de toonhoogte aanpassen volgens een bepaald patroon. Als voorbeeld om het concept te verduidelijken kan figuur \ref{figuur:fibo} dienen. In deze figuur wordt met behulp van de rij van Fibonacci een originele melodie getransformeerd tot een nieuwe. 

Er wordt getracht zo een transformatie te zoeken zodat de nieuw bekomen melodie nog steeds consonant is (nog steeds goed klinkt). Er zal ook onderzocht worden onder welke omstandigheden een transformatie al dan niet een consonante melodie oplevert. Om te bepalen of een nieuwe melodie goed klinkt is er ook nood aan een objectieve beoordeling van melodie\"en. Er wordt dus ook aandacht besteed aan het opstellen van een model dat hiervoor kan dienen. Een model dat zo goed mogelijk kan beoordelen of een gegeven melodie al dan niet, en in welke mate, consonant is. Het voordeel van het werken met transformaties in plaats van met generatie is dat men op deze manier nog een deel van de eigenheid van het originele werk kan behouden en het bekomen werk hierdoor ook minder artificieel zal overkomen bij de luisteraar.

\begin{figure}[!ht]
  \centering
  \includegraphics[width=0.5\textwidth]{0_Inleiding/fibo}
  \caption{Melodische transformatie m.b.v. rij van Fibonacci: Noten op de bovenste notenbalk liggen respectievelijk 1,1,2,3,5,8 halve tonen hoger dan op de onderste notenbalk.}
  \label{figuur:fibo}
\end{figure}

\section{Assumpties en beperkingen}
Op een onderdeel van hoofdstuk \ref{hoofdstuk:OBM} na, behandelt deze masterproef enkel muziek met precies \'e\'en melodielijn. Meerstemmige muziek waarbij meerdere noten tegelijkertijd gespeeld kunnen worden, wordt in deze thesis dus niet behandeld. Verder worden alle testen uitgevoerd op een corpus (Essencorpus \cite{url:essen}) bestaande uit muziekstukken van het folk genre.

Tot slot zal ook telkens wanneer een transformatie uitgevoerd is, er vanuit gegaan worden dat het originele muziekstuk telkens voldoet aan de algemene voorwaarden waaraan een muziekstuk moet voldoen. In dat geval kan de score van dit muziekstuk in het voorgestelde model telkens ook als referentie dienen voor de getransformeerde versie. 

\section{Overzicht van de tekst}
In het eerstvolgende hoofdstuk, hoofdstuk \ref{hoofdstuk:MA} wordt een korte inleiding gegeven tot de muziektheorie. Hierin worden enkel deze elementen behandeld die relevant zijn voor de rest van deze thesis. Vervolgens zal er een besproken worden hoe een bepaald muziekstuk objectief kan beoordeeld worden, dit zal gebeuren in hoofdstuk \ref{hoofdstuk:OBM}. In hoofdstuk \ref{hoofdstuk:MT} worden verschillende melodische transformaties toegelicht en met elkaar vergeleken. Vervolgens zal hoofdstuk \ref{hoofdstuk:ETT} twee algoritmes beschrijven. Deze algoritmes kunnen gebruikt worden om gegeven een aantal toegestane transformaties en een oorspronkelijke melodielijn, de meest waarschijnlijke getransformeerde melodielijn terug te geven. Hierna zullen een aantal experimenten, alsook hun resultaten besproken worden in hoofdstuk \ref{hoofdstuk:ER}. Tot slot wordt er in hoofdstuk \ref{hoofdstuk:B} nog teruggekeken op het geleverde onderzoek in een samenvattend besluit. Er wordt ook nog beschreven welk verder onderzoek zeker nog interessant zou kunnen zijn binnen dit onderwerp. 

%%% Local Variables: 
%%% mode: latex
%%% TeX-master: "masterproef"
%%% End: 

\newcommand{\signatuur}[2]{\ensuremath{%
  \vcenter{\offinterlineskip
    \halign{\hfil##\hfil\cr
            $\scriptstyle#1$\cr
            \noalign{\vskip1pt}
            $\scriptstyle#2$\cr}
  }}%
}

\chapter{Muzikale achtergrond}
\label{hoofdstuk:MA}

Aangezien deze thesis zal handelen over melodische transformaties wordt hieronder in het kort info gegeven over elementaire begrippen rond ritme en melodie van een muziekstuk en hoe deze voorgesteld kunnen worden. Voor lezers met een voorkennis in de muziekwereld zal dit onderdeel waarschijnlijk redundant zijn en kan er bijgevolg ook meteen overgegaan worden naar het volgende hoofdstuk. 

\section{Ritme}
In deze masterproef worden enkel melodische transformaties behandeld, waarbij het ritme ongewijzigd blijft. De tegenhangers hiervan zijn de ritmische transformaties\cite{thesis:thomas}, die in deze masterproef niet behandeld worden. Toch is het zeker nuttig om ook een zekere voorkennis te hebben van de betekenis van ritme in een muziekstuk. Melodie en ritme van een muziekstuk gaan namelijk hand in hand. Een basiskennis van ritmische begrippen zal dus zeker ook nuttig zijn voor het begrijpen van bepaalde melodische fenomenen.  

\subsection{Tijdssignaturen}
De tijdssignatuur geeft de ritmische structuur weer van het muziekstuk. Deze tijddssignatuur wordt weergegeven aan het begin van de notenbalk en ziet eruit als een breuk zonder streepje. Een voorbeeld hiervan is de vaak gebruikte \signatuur{3}{4} (of 'drie vierden'). In deze voorstelling staat het onderste getal voor welke nootlengte overeen komt met een tel. Het bovenste getal geeft weer hoeveel tellen in een maat voorkomen \cite{thesis:vincent}. In het voorbeeld van de signatuur \signatuur{3}{4} komt dit over met 4 tellen van lengte $\frac{1}{4}$. Voor de specifieke signatuur \signatuur{4}{4} (of 'vier vierden') heeft men nog een andere notatie, namelijk de letter C. Deze letter komt het woord \textit{common time}, aangezien deze signatuur zo typisch en veelvoorkomend is in moderne Westerse muziek. Figuur \ref{figuur:tijdssignatuur} geeft het gebruik van deze notatie weer. De figuur illustreert ook de 4 verschillende tellen van deze maatsoort.

\begin{figure}[!ht]
  \centering
  \includegraphics[width=0.5\textwidth]{1_Muzikale_Achtergrond/tijdssignatuur}
  \caption{Maat met voortekening van \textit{common time} tijdssignatuur.}
  \label{figuur:tijdssignatuur}
\end{figure}

\subsection{Tempo}
Een volgend belangrijk onderdeel dat het ritme van een muziekstuk bepaalt is het tempo. Het tempo van een stuk bepaalt namelijk de duur van de verschillende noten in het muziekstuk. Het temp wordt meestal uitgedrukt in aantal tellen per minuut. De lengte van een tel zelf wordt dan weer bepaald door de tijdssignatuur. Zo betekent bijvoorbeeld een tempo van 60 tellen voor een muziekstuk met signatuur \signatuur{3}{4} dat elke tel, ofwel elke kwartnoot, precies \'e\'en seconde duurt. 

\subsection{Nootlengtes}
De nootlengte is het laatste element dat de absolute duur van een noot zal bepalen. De nootlengte geeft de relatieve lengte van een noot weer ten opzichte van het tempo en de tijdssignatuur. De nootlengte wordt uitgedrukt door een breuk. Deze breuk heeft als relatieve lengte zijn verhouding tot de lengte van een tel. Zo zal een $\frac{1}{4}$-noot in \signatuur{3}{4} \'e\'en tel duren, en zal een $\frac{1}{8}$-noot in \signatuur{3}{4} twee tellen duren.

Een noot van hele lengte wordt aangeduid met een hol bolletje. Een noot van halve lengte wordt aangeduid met een half bolletje met een streep aan de rechterkant. Een $\frac{1}{4}$-noot wordt aangeduid zoals een halve noot maar dan met een vol bolletje. Een $\frac{1}{8}$-noot wordt voorgesteld door een vierde noot met een streepje vanboven. Vanaf een $\frac{1}{16}$-noot wordt er dan een streepje vanboven bijgezet telkens de nootlengte gehalveerd wordt. Deze notatie wordt ge\"illustreerd in figuur \ref{figuur:toonlengtes}. Door het gebruik van verbindingstekens(waarbij de nieuwe nootlengte de som is van de lengtes van de twee noten die verbonden zijn) kan men dan eender welke nootlengte bekomen die men maar wenst. 

\begin{figure}[!ht]
  \centering
  \includegraphics[width=0.5\textwidth]{1_Muzikale_Achtergrond/nootlengtes}
  \caption{Illustratie van een aantal veelvoorkomende nootlengtes.}
  \label{figuur:nootlengtes}
\end{figure}

\section{Melodie}
Naast ritme is het andere fundamentele bestanddeel van een muziekstuk de melodie. Waar het ritme de structuur van een muziekstuk weergeeft, is de melodie het voornaamste bestanddeel van de muziek dat een gevoel meegeeft aan het stuk. De melodie geeft een toonhoogte of freqentie mee aan elke noot. Aangezien deze thesis handelt over melodische transformaties is het van belang te weten wat het begrip ``melodie'' precies inhoudt. Het is namelijk dat gedeelte van een muziekstuk waarop een transformatie zal toegepast worden. Het ritme van een muziekstuk wordt in deze thesis onveranderd gelaten.

\subsection{Toonhoogte}
Toonhoogte kan in het algemeen op twee manieren voorgesteld worden. Een eerste manier is fysisch waarbij elke toon met een bepaalde frequentie overeenkomt, een andere manier is meer muziek-theoretisch, waarbij de toonhoogte als discreet beschouwd wordt. 

In deze masterproef zal met de laatste voorstellingswijze gewerkt worden. Dit omdat we noten willen transformeren naar nieuwe noten en niet frequenties naar nieuwe frequenties (die dan niet overeen komen met een noot en bijgevolg zeer waarschijnlijk niet goed gaan klinken in het geheel). Deze toonhoogte wordt dan voorgesteld door een naam. Er zijn twee naamgevingen die vaak gebruikt worden. Eerst is er de naamgeving waarbij de letters A t.e.m. G gebruikt worden. De andere naamgeving maakt gebruik van de woorden do -- re -- mi -- fa -- sol -- la -- si, de relatie tussen deze 2 naamgevingen wordt weergegeven in tabel \ref{tabel:toonhoogte}. Noten kunnen uiteraard ook op een notenbalk weergegeven worden, zie figuur \ref{figuur:toonladder} voor een visuele weergave. Hoe hoger de noot op de notenbalk, hoe hoger haar frequentie.

\begin{table}
  \centering
  \begin{tabular}{ c c c c c c c }
    A & B & C & D & E & F & G \\
    \hline
    \hline
    la & si & do & re & mi & fa & sol \\
  \end{tabular}
  \caption{Opsomming van de toonhoogtes in 2 verschillende benamingen.}
  \label{tabel:toonhoogte}
\end{table}

\begin{figure}[!ht]
  \centering
  \includegraphics[width=0.5\textwidth]{1_Muzikale_Achtergrond/toonladder}
  \caption{Noten do t.e.m. si op een notenbalk.}
  \label{figuur:toonladder}
\end{figure}

\subsection{Octaaf en onderverdeling in toonhoogtes}
Een eenvoudige definitie van een octaaf is het interval tussen een gegeven toonhoogte en de toonhoogte met het dubbele van de frequentie van de eerste. Deze noten worden als zeer gelijkaardig ervaren en krijgen daarom dezefde benaming (bijvoorbeeld beide een A). Hierdoor gaat men vaak in muzieknotatie wanneer men een bepaalde noot benoemt, ook aangeven tot welk octaaf de noot behoort (want een bepaalde noot komt met meerdere toonhoogtes overeen). Dit doet men door in subscript de index van het octaaf weer te geven, een A (of la) in het vierde octaaf wordt dan weergegeven door $A_{4}$.

Een octaaf wordt opgedeeld in 12 toonhoogtes. De afstand tussen elk paar van 2 opeenvolgende toonhoogtes wordt een halve toon genoemd. Hiervan zijn er slechts 7 tonen die ook deel uitmaken van de toonladder. De andere 5 noten worden voorgesteld op de notenbalk relatief ten opzichte van een van de nabijgelegen tonen die slechts een halve toon hier vanaf ligt door het gebruik van een kruis ($\sharp$) of een mol($\flat$). Een kruis dient om aan te duiden dat de noot die bedoeld wordt een halve toon hoger is dan de noot die ervoor staat. Een mol gebruikt men om aan te duiden dat de noot die bedoeld wordt een halve toon lager is dan de noot die net voor het molteken staat.

\subsection{Toonladders en toonaarden}
Zoals reeds vermeld werd in het vorige deel, bestaat een octaaf uit 12 tonen waarvan er slechts 7 tot de eigenlijke toonladder behoren. De noten binnen een toonladder worden bepaald door de intervallen tussen de noten. In het algemeen zijn er twee grote onderverdelingen om een toonladder te construeren. 

Een eerste resultaat wordt de ``grote toonladder'' genoemd. deze wordt bepaald door de opeenvolging van stappen 1-1-$\frac{1}{2}$-1-1-1-$\frac{1}{2}$. Het meeste elementaire voorbeeld van een toonladder die hieraan voldoet is de toonladder van ``Do-Groot'' waarbij volgende noten voldoen aan de opgegeven intervalafstanden: ``do - re - mi - fa - sol - la - si - do'', zoals weergegeven in figuur \ref{figuur:do_groot}. 

\begin{figure}[!ht]
  \centering
  \includegraphics[width=0.5\textwidth]{1_Muzikale_Achtergrond/do_groot}
  \caption{Toonladder van ``Do Groot''}
  \label{figuur:do_groot}
\end{figure}

Een tweede mogelijkheid is de zogenaamde ``kleine toonladder''. In deze toonladder komen de intervalafstanden overeen met 1-$\frac{1}{2}$-1-1-$\frac{1}{2}$-1-1. Een concreet voorbeeld hiervan is de toonladder van ``La-Klein'' waarbij deze noten voldoen aan de voorwaarden: ``la - si - do - re - mi - fa - sol - la'', deze wordt weergegeven in figuur \ref{figuur:la_klein}. 

\begin{figure}[!ht]
  \centering
  \includegraphics[width=0.5\textwidth]{1_Muzikale_Achtergrond/la_klein}
  \caption{Toonladder van ``La Klein''}
  \label{figuur:la_klein}
\end{figure}

Deze toonladder is de kleine versie van de toonladder van Do groot, aangezien ze dezelfde noten gebruikt, de toonladder is als het ware een verschuiving van die van Do groot. Het verschil tussen beide toonladders is dus de functie van elke noot in de bijhorende toonaarden. Iets wiskundiger verwoord is er ook een verschil van frequentie waarin bepaalde noten voorkomen in muziekstukken horende bij een kleine of grote toonladder \cite{book:musicAndProbability}. Van de kleine toonladders zijn ook nog een harmonische en melodische versie, die nog andere afstanden hanteren, maar aangezien deze niet zo vaak gebruikt worden, zou het te ver leiden deze ook te bespreken.

Het feit of een toonladder groot of klein is gaat vaak ook de sfeer bepalen van het muziekstukje. Zo zal een muziekstuk dat geschreven is in een grote toonladder eerder een vrolijk karakter hebben. Dit terwijl een stukje dat geschreven is in een kleine toonladder eerder een droeviger karakter zal hebben. 

Niet alleen de toonladder zelf, ook de noten in de toonladder hebben hun functie. Zo is bijvoorbeeld de eerste noot van een toonladder een rustpunt waarop het muziekstuk vaak be\"eindigd wordt. De vijfde noot wordt de dominant genoemd en is ook sterk vertegenwoordigd in het muziekstuk. Deze nooit cre\"eert namelijk spanning. Vaak wordt deze spanning dan ook opgelost door een overgang naar de eerste noot, als rustpunt.


%%% Local Variables: 
%%% mode: latex
%%% TeX-master: "masterproef"
%%% End: 

\chapter{Objectieve Beoordeling Muziekstuk}
\label{hoofdstuk:OBM}
Om melodische transformaties te kunnen beoordelen is er nood aan een framework dat kan voorspellen of een gegeven melodie al dan niet goed klinkt. In dit hoofdstuk worden twee zo een modellen besproken die een bepaalde `consonantiescore' (maat voor het goed klinken van een muziekstuk) gaan toekennen aan een muziekstuk. Allereerst wordt een aanpak besproken die gebruik maakt van een artifici\"eel neuraal netwerk \cite{book:ANN}. Deze methode is speciaal gemaakt voor het beoordelen van polyfone muziek, waarbij er meerdere noten tegelijk klinken. Polyfone muziek wordt in verdere hoofdstukken van deze masterproef niet meer besproken. Het eerste onderdeel van dit hoofdstuk zal bijgevolg dus weinig invloed hebben op het vervolg van de tekst. De methode is echter wel onderzocht geweest en leverde enkele interessante resultaten op waardoor de methode toch even in het kort vermeld wordt. Als tweede methode wordt een zogenaamd `RPK-Model' besproken dat gebaseerd is op een gelijknamig model uit het boek `Music and Probability' \cite{book:musicAndProbability} van David Temperley \cite{url:temperley}. Dit model werkt op muziek met een enkele melodielijn en zal ook het model zijn dat als verificator gebruikt zal worden bij de experimenten in het vervolg van deze masterproef.

\section{Neuraal Netwerk}
\label{OBM:NN}
In het boek `Musimathics' \cite{book:musimathics} wordt er een gedeelte gewijd aan het meten van de consonantie van akkoorden (samenklank van meerdere noten). Dit gebeurt kort samengevat door het trainen van een neuraal netwerk op zogenaamde tweeklanken (twee noten die tegelijkertijd gespeeld worden). Het neuraal netwerk maakt dan de veralgemening naar hogere orde samenklanken. Het neuraal netwerk dat gemaakt werd als verificator bestond uit drie lagen. een invoerlaag, een uitvoerlaag en dan nog een verborgen laag. De invoerlaag bestond uit 12 neurons die elk voor een van de 12 verschillende noten staan. Twee noten die een octaaf uit elkaar liggen (en muziektechnisch dezelfde naam krijgen) zullen dus ook op eenzelfde neuron afgebeeld worden. Deze 12 input neurons krijgen een `1' als input wanneer er een noot in het akkoord zit dat afgebeeld wordt op die neuron. In het andere geval krijgt deze een `0' als input. Vervolgens komen we op een hidden layer terecht die volledige geconnecteerd is en uit 6 neurons bestaat. Tot slot komen we nog bij de output layer uit die uit slechts 1 neuron bestaat. Deze neuron gaat een getal tussen 0 en 1 teruggeven, hoe dichter het getal bij 1 ligt hoe zekerder het neuraal netwerk is dat het ingegeven akkoord consonant is. Hoe dichte bij 0 hoe zekerder hij is dat het akkoord dissonant (tegengestelde van consonant) is.\\
Dit netwerk werd getraind op alle mogelijke tweeklanken gebruik makend van de \textit{backpropagation of error}-methode\cite{url:backpropagation}. Voor elke mogelijke tweeklank werd afhankelijk van de afstand (in halve tonen) tussen de twee noten bepaald of ze goed samen klinken of niet. Dit volgens de regels van de muziektheorie. Deze regels komen ook overeen met de verhoudingen van de frequenties van de twee invoernoten. Hoe kleiner de getallen in de vereenvoudigde breuk van de frequenties, hoe beter het akkoord klinkt. Na het trainen van het neuraal netwerk kan dit gebruikt worden om meerstemmige muziek te gaan beoordelen. Op elk tijdstip waarop er noten gespeeld worden kan men deze noten als input in het neuraal netwerk steken. De uitvoer van het netwerk geeft dan een maat voor de consonantie terug. Als we dit doen voor elk tijdstip in het muziekstuk waarop er noten gespeeld worden en we nemen dan het gemiddelde over al deze tijdstippen krijgen we een maat voor de `gemiddelde consonantie' van het gehele muziekstuk.\\
Nadat het neuraal netwerk getraind werd, bleek het neuraal netwerk goed te generaliseren naar akkoorden met meer dan 2 noten. Dit wil zeggen dat akkoorden van meer dan twee noten die muziektheoretisch `goed' zouden moeten klinken ook door het netwerk als dusdanig beoordeeld werden. Het netwerk kon dus dienen als een soort van alternatieve voorstelling van de regels van de muziektheorie wat de samenklank van akkoorden betreft. Het probleem lag er echter in dat de klassieke werken van Bach \cite{url:bach} en Mozart \cite{url:mozart} waarop getest werd zelf helemaal niet zo consonant waren als oorspronkelijk gedacht. En dit in die mate dat vaak tot een derde van de akkoorden in zo een stuk als dissonant (tegengestelde van consonant) bestempeld worden. Deze akkoorden blijken ook daadwerkelijk dissonant te zijn. Het is slechts door de context, de noten die net voor het akkoord gespeeld worden, dat deze akkoorden in die stukken toch niet als dissonant ervaren worden door de luisteraar. Het neuraal netwerk dat hier gebruikt werd houdt echter geen rekening met deze context. Dit leidt er wel toe dat dit neuraal netwerk niet gebruikt kan worden als verificator, aangezien zelfs muziekstukken van Bach en Mozart door de verificator niet als consonant herkend worden. Het is daarentegen wel nog maar eens een bevestiging van het genie van componisten als Bach en Mozart, de kunst ligt het niet in het volgen van de regels maar in het weten wanneer en hoe de regels gebroken mogen worden.\\
De broncode die geschreven werd voor het trainen van dit neuraal netwerk is beschikbaar in appendix \ref{Broncode:ANN}. Er zijn een aantal parameters die ingesteld kunnen worden in dit algoritme. Er is allereerst het aantal neuronen in de verborgen laag. Er kan ook een waarde voor de `learning rate' opgegeven worden en dan zijn er nog twee parameters die een bias kunnen defini\"eren.

\section{Keuze tussen Polyfone en Monofone Muziek}
\label{OBM:OMM}
In deze thesis wordt gewerkt met transformaties op een muziektheoretische manier (en dus niet fysisch met frequenties). Daarom is wordt er gebruik gemaakt van muziekstukken die in het MusicXML\cite{url:musicxml} formaat beschikbaar zijn. Het nadeel van deze keuze is dat er slechts een beperkt aantal muziekstukken beschikbaar is. Dit in tegenstelling tot bijvoorbeeld het MIDI-formaat\cite{url:midi} waarbij dit niet het geval is. De polyfone(meerstemmige) muziek die beschkbaar is in het juiste formaat bestaat eigenlijk bijna uitsluitend uit klassieke muziek. Verder is er ook nog een redelijk groot corpus beschikbaar van folkmuziek, het zogenaamde Essencorpus\cite{url:essen}. Deze muziek is wel monofoon (eenstemmig). Aangezien de methode met het neuraal netwerk uitgelegd in bovenstaand onderdeel niet voldoende werkte voor de meerstemminge stukken moest er een andere weg ingeslagen worden. Ofwel verder werken met meerstemmige muziek maar met een ander framework dat de context in rekening brengt. Ofwel de focus verleggen naar eenstemmige muziek. Gezien het grootste deel van de meerstemmige muziekstukken klassieke stukken zijn, is de keuze gemaakt om over te schakelen op eenstemmige muziek. Dit omdat de complexiteit van het aanpassen van muziekstukken met meerdere lijnen veel hoger is dan die van muziekstukken met slechts een instrument. Ook omdat klassieke muziekstukken veel delicater zijn om te behandelen dan stukken folkmuziek die enkel uit een melodielijn bestaan. Alle transformaties die zullen besproken worden in de rest van deze masterproef zullen dus ook uit een enkele melodielijn bestaan.

\section{RPK-Model}
\label{OBM:RPK}
Aangezien er nood was aan een framework voor het beoordelen van muziekstukken van slechts een enkele melodielijn werd er uiteindelijk uitgekomen bij het zogenaamde RPK-model. Dit model wordt beschreven in het boek `Music and Probability'\cite{book:musicAndProbability}. Het model dat besproken gaat worden in dit onderdeel en dus ook gebruikt werd in de rest van het onderzoek is gebaseerd op het model uit dit boek. Er werden een aantal vereenvoudigingen gedaan gebaseerd op extra data die beschikbaar is in het MusicXML formaat. Dit RKP-model gaat dus uit van een enkele lijn melodie (er wordt slechts een noot tegelijkertijd gespeeld). De waarschijnlijkheid van een bepaalde noot in een muziekstuk wordt gekenmerkt door de combinatie van 3 kenmerken waar het model naar genoemd is. Allereerst is er de zogenaamde \textit{range}, dit is de afwijking tot een centrale toonhoogte. Deze centrale toonhoogte is de noot die in het midden ligt van de distributie van alle noten uit alle muziekstukken van het beschikbare corpus van folkmuziek. Globaal gezien hebben noten die dicht bij die centrum liggen een grotere waarschijnlijkheid tot voorkomen als noten die verder van dit centrum gelegen zijn. Dit wordt ge\"illustreerd in figuur \ref{figuur:range}, waarbij elke noot voorgesteld wordt door een geheel getal. De centrale C (do) heeft als waarde 60 gekregen. Een eenheid in deze schaal komt overeen met een halve toon, de volgende C krijgt dus als waarde 72 omdat het verschil tussen de deze twee noten 12 halve tonen is. Verder hebben we ook nog de \textit{proximity}, dit heeft te maken met de relatieve afstand tot de voorgaande noot. Bepaalde groottes van sprongen zijn veel waarschijnlijker dan andere. Een sprong met een terts of een kwint komt bijvoorbeeld veel vaker voor dan een sprong van een sext. In het algemeen zijn kleine sprongen veel waarschijnlijker dan grotere. Figuur \ref{figuur:proximity} geeft de frequenties weer waarmee bepaalde intervallen tussen opeenvolgende noten voorkomen in het Essencorpus. Ten slotte wordt er ook nog rekening gehouden met de \textit{key}, oftewel de toonaard van het muziekstuk. Afhankelijk van de toonaard zijn bepaalde noten waarschijnlijker om voor te komen dan andere. Zo is de grondtoon van een toonladder bijvoorbeeld altijd sterk aanwezig terwijl de noot die een halve toon hoger ligt quasi nooit zal voorkomen in het muziekstuk. Ook is er een verschil in profiel tussen majeur en mineur toonaarden, hier wordt ook rekening mee gehouden. Dit wordt ook ge\"illustreed in figuren \ref{figuur:key_major} en \ref{figuur:key_minor} die resprectievelijk voor stukken die in grote en kleine toonaarden staan de distributies van noten uit het Essencorpus weergeeft. De \textit{range}- en \textit{proximity}-waarden worden gemodelleerd door een normaalverdeling rond respectievelijk de centrale en de vorige noot uit het muziekstuk. De \textit{key}-waarde van een noot wordt bepaald aan de hand van de frequentie van voorkomen van deze zijn nootfunctie in alle muziekstukken van het corpus.

\begin{figure}[!ht]
  \centering
  \includegraphics[width=0.75\textwidth]{2_Objectieve_Beoordeling/range}
  \caption{Distributie van alle noten in het Essencorpus.}
  \label{figuur:range}
\end{figure}

\begin{figure}[!ht]
  \centering
  \includegraphics[width=0.75\textwidth]{2_Objectieve_Beoordeling/proximity}
  \caption{proporties van voorkomen van alle intervallen van opeenvolgende noten in het Essencorpus.}
  \label{figuur:proximity}
\end{figure}

\begin{figure}[!ht]
  \centering
  \includegraphics[width=0.75\textwidth]{2_Objectieve_Beoordeling/key_major}
  \caption{proporties van voorkomen in het Essencorpus van nootfuncties in grote toonladder.}
  \label{figuur:key_major}
\end{figure}

\begin{figure}[!ht]
  \centering
  \includegraphics[width=0.75\textwidth]{2_Objectieve_Beoordeling/key_minor}
  \caption{proporties van voorkomen in het Essencorpus van nootfuncties in kleine toonladder.}
  \label{figuur:key_minor}
\end{figure}

%%% Local Variables: 
%%% mode: latex
%%% TeX-master: "masterproef"
%%% End: 

\chapter{Melodische Transformatie}
\label{hoofdstuk:MT}

\section{Inleiding}
In dit hoofdstuk zullen verschillende melodische transformaties besproken worden met hun voor- en nadelen. Deze transformaties zullen werken op melodielijnen die monofoon zijn. Deze melodielijnen zullen behandeld worden als een opeenvolging van noten (toonhoogtes) en het ritme zal na transformatie gewoon behouden blijven. Enkel de toonhoogte van een noot zal aangepast worden. Het ritme zal bij de transformaties die hier besproken staan ook geen invloed hebben op de transformatie van een noot. 

\subsection{Fibonacci}
De transformaties die hier als voorbeeld gegeven worden zullen vaak in een zekere vorm de rij van Fibonacci\cite{url:Fibonacci} bevatten. Dit komt omdat in een vroeg stadium van het onderzoek de vraag bestond of transformaties die voortgaan op de rij van Fibonacci een beter resultaat zouden bieden dan willekeurige transformaties. Dit omdat de rij van Fibonacci in de natuur zo nadrukkelijk aanwezig is dat ook redelijkerwijs de vraag gesteld zou kunnen worden of ook in de muziekwereld zo een invloed zou kunnen hebben. Dit bleek niet het geval te zijn. Maar aangezien het ook geen slechtere resultaten gaf en omdat het ook onderdeel van het onderzoek was, zijn deze transformaties gebaseerd op de rij van Fibonacci vaak gebruikt ter illustratie.

\subsection{Beschrijving van een Transformatie}
De melodische transformaties die besproken worden in dit hoofdstuk gaan telkens beschreven worden aan de hand van een tabel. Deze tabel zal telkens een mapping van 8 waarden bevatten. de tweede rij zal altijd waarden tussen -5 en +6 bevatten die als betekenis hebben met welke waarde (namelijk de hoeveelheid halve tonen) een bepaalde noot verhoogd moet worden. Welke noot met welke hoeveelheid getransformeerd moet worden wordt dan telkens weergegeven via een waarde op de bovenste rij. Deze rij is ook telkens cyclisch modulo 8. Wanneer bijvoorbeeld zoals in tabel \ref{tabel:transformatie1} de index van de noot weergegeven wordt op de bovenste rij, dan zal de noot op positie 4 met 5 halve tonen verhoogd worden door de transformatie. Maar ook bijvoorbeeld de noot op positie 12 zal met 5 halve tonen verhoogd worden omdat 12 ook 4 geeft als rest na deling door 8.

\subsubsection{Afronding naar de toonaard}
\label{sub:afronding}
In hoofdstuk \ref{OBM:RPK} werd reeds het RPK-model besproken. Dit model wordt gebruikt ter evaluatie van de transformaties. Bij de bespreking van dit model was een van de drie belangrijke kenmerken van een noot om de probabiliteit te bepalen de \textit{key}. Noten die in de toonaard voorkomen zijn zo veel waarschijnlijker om voor te komen dan noten die niet in de toonaard voorkomen dat ervoor gekozen is om na uitvoer van de transformaties nog een zogenaamde `afronding' door te voeren. Deze afronding bestaat erin om na de transformatie van een noot in de melodielijn, indien deze niet tot de toonaard behoort, te verhogen of verlagen met een halve toon naar die noot van de twee die de hoogste probabiliteit heeft volgens het RPK-model. Deze twee noten zullen ook telkens beide wel tot de toonaard behoren. De transformaties zelf zullen dus theoretisch geen rekening houden met deze afronding, met andere woorden, dit zal niet expliciet vermeld worden in de beschrijvingen van de transformaties. Het is echter wel belangrijk te weten dat ik wel degelijk altijd gebeurt. Vandaar dat het soms ook kan zijn dat het lijkt alsof een transformatie een halve toon the hoog of te laag is uitgevoerd voor een bepaalde noot. Dit ligt dan enkel aan de zonet besproken afronding.

\section{Afbeelding afhankelijk van positie}
\label{MT:positie}
\subsection{Beschrijving transformatie}
Een eerste zeer voor de hand liggende melodische transformatie is er eentje die een noot in een muziekstuk gaat transformeren enkel naargelang zijn positie in de melodielijn. De transformatie die ter illustratie dient van dit concept wordt beschreven in tabel \ref{tabel:transformatie1}. Zo zal de noot op positie 6 door deze transformatie met 1 halve toon verhoogd worden. De noot op positie 13 zal met 4 halve tonen verlaagd worden. 

\begin{table}
  \centering
  \begin{tabular}{c | c c c c c c c c }
    Index (mod 8) & 0 & 1 & 2 & 3 & 4 & 5 & 6 & 7 \\
    \hline
    \hline
    Verhoging & 1 & 1 & 2 & 3 & 5 & -4 & 1 & -3 \\
  \end{tabular}
  \caption{Transformatie afhankelijk van de positie van de noot.}
  \label{tabel:transformatie1}
\end{table}

\subsubsection{Voorbeeld}
In figuur \ref{figuur:voorbeeld_transformatie_1} wordt ter illustratie deze transformatie toegepast op een korte melodielijn. De bovenste lijn geeft de originele melodie weer, de onderste lijn geeft het getransformeerde resultaat weer. Bij de twee melodielijnen staat bij elke noot telkens ook zijn geheel getal representatie in halve noten (modulo 12). Dit zodat het voor de lezer makkelijker na te gaan is hoe de transformatie precies verloopt. Tussen de twee notenbalken wordt dan aangegeven welke sprong de transformatie oplegt op de melodielijn. Zo zal het verschil in getalwaarde tussen overeenkomstige noten op de bovenste en de onderste notenbalk telkens gelijk zijn aan de waar die hier aangegeven staat. Merk op dat die voor de tweede en zevende noot in dit voorbeeld niet het geval is. Hier lijkt het verschil tussen de noten in de bovenste en onderste lijn telkens een halve toon kleiner dan het zou moeten zijn. Dit komt door de afronding besproken in onderdeel \ref{sub:afronding} aangezien de noot met getalwaarde 8 (G$\sharp$/A$\flat$) niet tot de toonaard van het muziekstuk (Do groot) behoort.

\begin{figure}[!ht]
  \centering
  \includegraphics[width=0.75\textwidth]{3_Melodische_Transformatie/transfo1}
  \caption{Voorbeeld van toepassing transformatie afhankelijk van positie.}
  \label{figuur:voorbeeld_transformatie_1}
\end{figure}

\subsubsection{Bespreking transformatie}
Het grootste voordeel van deze transformatie is dat hij zeer eenvoudig uit te voeren en zeer elementair is. Enkel de positie van de noot is van belang met betrekking tot naar welke noot hij getransformeerd zal worden. Een groot nadeel van deze transformatie is dus ook dat deze totaal geen rekening houdt met de eigenlijke toonhoogte van de noot die getransformeerd gaat worden. Er wordt ook totaal geen rekening gehouden met de context van de noot. En als er in het originele stuk patronen zitten zullen die ook nooit herkend en gelijk getransformeerd worden tenzij in het zeer specifieke geval dat deze telkens mooi op een veelvoud van 8 noten van elkaar voorkomen. Deze transformatie zal dus ook verder niet veel gebruikt worden. Het kan wel interessant zijn om deze transformatie te zien als een soort van baseline voor een willekeurige transformatie om dan andere transformaties mee te kunnen vergelijken. Ook kan het interessant zijn na te gaan of er veel verschil zit tussen zo een transformaties waarin grote sprongen zitten ten opzichte van transformaties waarin vooral kleinere sprongen zitten (aangezien deze het originele stuk minder zullen gaan vervormen).

\section{Afbeelding afhankelijk van afstand ten opzichte van vorige noot}
\label{MT:afstand_vorige}
\subsubsection{Beschrijving transformatie}
Een andere melodische transformatie is er een die een noot in een muziekstuk gaat transformeren naargelang zijn afstand in ten opzichte van de vorige noot. Hierbij zal er gekeken worden naar de afstand van de noot op positie `x' in de originele melodielijn tot de noot op positie `x-1' in de nieuwe melodielijn (dit is dus de getransformeerde waarde van de vorige noot in het originele muziekstuk). De eerste noot van eender welk muziekstuk wordt in deze transformatie behouden omdat deze noot geen voorgaande noot heeft. Wat ook nog speciaal is aan deze transformatie is dat de verhoging die weergegeven wordt in de transformatietabel toegepast wordt in de tegengestelde richting van waar de huidige noot staat ten opzichte van de vorige. De transformatie die ter illustratie dient van dit concept wordt beschreven in tabel \ref{tabel:transformatie2}. Zo zal een noot die 2 halve tonen hoger ligt dan de getransformeerde waarde van de vorige noot met 1 halve toon verlaagd worden. Een noot die 6 halve tonen lager ligt dan de getransformeerde waarde van de vorige noot zal met 2 tonen verhoogd worden. En zal een noot die 1 halve toon lager ligt dan de getransformeerde toonhoogte van de vorige noot nog eens met 4 halve tonen verder verlaagd worden (omdat de waarde in de tabel negatief is).

\begin{table}
  \centering
  \begin{tabular}{c | c c c c c c c c }
    Diff (mod 8) & 0 & 1 & 2 & 3 & 4 & 5 & 6 & 7 \\
    \hline
    \hline
    Verhoging & 5 & -4 & 1 & -3 & 1 & 1 & 2 & 3 \\
  \end{tabular}
  \caption{Transformatie afhankelijk van de afstand van de huidige noot tot de vorige noot.}
  \label{tabel:transformatie2}
\end{table}

\subsubsection{Voorbeeld}
In figuur \ref{figuur:voorbeeld_transformatie_2} wordt een voorbeeld gegeven van een korte melodielijn waarop deze transformatie wordt toegepast. Voor de eerste noot wordt er geen transformatie weergegeven, dat is omdat deze ook geen vorige noot heeft en dus niet getransformeerd wordt. Als we dan bijvoorbeeld naar de tweede noot kijken dan heeft deze getalwaarde 5, de vorige noot uit de getransformeerde melodie heeft waarde 7 dus het verschil is -2. De absolute waarde is 2 waardoor de sprong als grootte -4 en aangezien het verschil negatief is moet deze waarde bij die van de noot opgeteld worden (want we willen een sprong in de richting van de vorige noot, al zal in dit geval toch de andere richting uit gegaan worden aangezien de noot in de tabel zelf negatief is). Zo komen we normaal gezien uit op een noot met getalwaarde 1. Deze noot ligt niet in de toonaard en omdat de noot met getalwaarde 0 (die de grondtoon is van de toonaard) waarschijnlijker is dan die met waarde 2 wordt de noot met waarde 0 als getransformeerde noot gekozen. Als we dan verder gaan naar de volgende noot merken we dat het verschil 4 is. Een absolute waarde van 4 voor het verschil geeft aanleiding tot een sprong van grootte 1. aangezien het verschil positief is moet de sprong echter in tegengestelde richting uitgevoerd worden en zal de sprong als waarde -1 hebben. En zo komen we na afronding bij een noot met getalwaarde 4 uit.

\begin{figure}[!ht]
  \centering
  \includegraphics[width=0.75\textwidth]{3_Melodische_Transformatie/transfo2}
  \caption{Voorbeeld van toepassing transformatie afhankelijk van interval ten opzichte van vorige noot.}
  \label{figuur:voorbeeld_transformatie_2}
\end{figure}

\subsubsection{Bespreking transformatie}
Het interessante aan de transformatie die hier beschreven werd is dat deze de noten niet als losstaand beschouwt, maar ook de context gedeeltelijk in rekening brengt. Dit betekent dat eenzelfde noot naar zeer veel verschillende noten kan getransformeerd worden afhankelijk van de voorgaande noot. Een voordeel van deze transformatie is ook dat als er een zekere repetitiviteit in het originele muziekstuk zit, deze bijna altijd ook in het getransformeerde stuk zal voorkomen (enkel de noot die net voor zo een patroon voorkomt kan nog een extra invloed hebben). Dit maakt dat de structuur van het originele stuk nog iets meer bewaard blijft. Een nadeel van deze transformatie is dat deze wel nog steeds blind is voor de rest van de context.  

%%% Local Variables: 
%%% mode: latex
%%% TeX-master: "masterproef"
%%% End: 

\chapter{Transformaties combineren}
\label{hoofdstuk:ETT}

In dit hoofdstuk worden twee algoritmes behandeld die varianten zijn van elkaar. Beide algoritmes transformeren een gegeven muziekstuk tot een nieuw muziekstuk zodat dat de totale consonantiescore zo hoog mogelijk is. Dit gebeurt door een aantal toegelaten operaties op het oorspronkelijke muziekstuk. Buiten de originele melodielijn zullen ook een aantal verschillende transformaties (zoals die beschreven zijn in onderdeel \ref{MT:afstand_vorige}) meegegeven worden aan het algoritme. Deze transformaties zullen door het algoritme gebruikt mogen worden om het originele stuk te transformeren naar een nieuw stuk met een hogere consonantiescore (De score die gebruikt wordt, is deze beschreven in onderdeel \ref{OBM:RPK}, volgens het `RPK-Model').

Voor elke noot van de melodielijn heeft het algoritme de keuze om ofwel de noot te behouden, ofwel deze te transformeren conform een van de transformaties die meegegeven werd aan het algoritme. Dit algoritme wordt beschreven in onderdeel \ref{ETT:algo1}. 

In gedeelte \ref{ETT:algo2} wordt een algoritme beschreven dat hetzelfde doel heeft maar voldoet aan een extra voorwaarde: een transformatie mag enkel toegepast worden indien dit gebeurt op een minimum aantal opeenvolgende noten van het muziekstuk. Deze extra voorwaarde gaat het \textit{overfitten} tegen, indien deze er niet zou zijn convergeert het algoritme te snel naar een uitgevlakte melodielijn.

\subsubsection{Opmerking}
Een eerste doel van deze algoritmen is om te deze gaan gebruiken om muziekstukken te gaan cre\"eren die goed zullen klinken. Er wordt nadrukkelijk geprobeerd om die melodielijn te vinden waarnaar getransformeerd kan worden die een zo hoog mogelijke consonantiescore heeft volgens het RPK-model. Wat het algoritme zal doen door de consonantie te verhogen, is ervoor zorgen dat het bekomen muziekstuk normaal gezien niet slecht zou mogen klinken aangezien verondersteld wordt dat het origineel goed klinkt en de score enkel verhoogt wordt. Er werd echter reeds aangehaald dat muziekstukken met een zeer hoge consonantiescore vaak niet interessant klinken (veel dezelfde noten en heel kleine sprongen tussen opeenvolgende noten). Het is ook belangrijk dit in het achterhoofd te houden. 

Een tweede opzet van deze algoritmen is om achteraf te kunnen bepalen hoe afhankelijk de consonantiescore zal zijn van het aantal transformaties dat toegepast wordt op het muziekstuk, wat de eventuele minimum transformatielengte als invloed heeft en hoe het aantal beschikbare transformaties de consonantiescore bepaalt.

\section{Beste sequentie}
\label{ETT:algo1}

\subsection{Doelstelling}
Het algoritme dat hier beschreven wordt is afhankelijk van twee parameters. De eerste parameter is een originele melodielijn. De tweede parameter is een verzameling van transformaties (gedefinieerd zoals in onderdeel \ref{MT:afstand_vorige}), die gebruikt mogen worden door het algoritme. Het doel is nu om aan de hand van enkel deze transformaties de originele melodielijn te transformeren tot een nieuwe melodielijn met een zo hoog mogelijke consonantiescore. Dit moet gebeuren in een enkele \textit{pass} over de melodielijn. Hierbij heeft het algoritme voor elke noot in het muziekstuk twee mogelijke keuzes.
 
Een eerste mogelijkheid is dat de noot niet getransformeerd wordt en identiek overgenomen wordt in de nieuwe melodielijn. De tweede mogelijkheid bestaat erin dat de noot getransformeerd mag worden, maar dan enkel volgens een van de beschikbare transformaties.

\subsection{Idee van het algoritme}
\subsubsection{Bereik van een transformatie}
Een belangrijke observatie is dat eender welke noot in het muziekstukje zelf op slechts maximaal 14 verschillende noten kan afgebeeld worden. Elke noot kan namelijk door transformatie enkel afgebeeld worden op een noot die maximaal 5 halve tonen lager ligt dan de oorspronkelijke noot en ook maximaal 6 halve tonen hoger ligt dan de originele noot (dit is een deel van de definitie van de soort transformaties die gebruikt worden in het algoritme, zie hoofdstuk \ref{MT:afstand_vorige}).

Er is echter nog een speciaal geval waarbij een transformatie tot een van de twee extreme sprongen zou leiden (dus exact -5 of +6) en deze noot dan ook nog eens geen deel zou uitmaken van de toonaard. In dat geval is het mogelijk dat de noot nog een halve toon verder afgerond wordt om terug tot op een noot te komen die in de toonaard ligt.

Elke noot kan dus theoretisch gezien (na afronding) afgebeeld worden op eender welke noot die maximaal 6 halve tonen lager en maximaal 7 halve tonen hoger ligt dat zichzelf. Dit zijn in totaal 14 verschillende mogelijkheden.

\subsubsection{Hoog niveau idee van het algoritme}
Het belangrijkste idee van het algoritme is gebaseerd op de principes van \textit{dynamic programming} \cite{url:DP}. Toegepast op dit algoritme komt het er op neer dat achtereenvolgens voor elke noot in het muziekstuk het beste pad (en bijhorende beste score) bijgehouden zal worden voor elk van de 14 mogelijke toonhoogtes die deze noot kan aannemen in de nieuwe melodielijn. En enkel op deze optimale paden zal verder gerekend worden om te bepalen wat de beste paden zijn tot de 14 mogelijke toonhoogtes die de volgende noot in het muziekstuk kan aannemen. Dit wordt dan verder herhaald tot alle noten van de originele melodielijn overlopen zijn.

\subsection{Werking van het algoritme}
In dit onderdeel zal de werking van het algoritme beschreven worden. Als extra referentie voor de lezer is de broncode bijgevoegd in appendix \ref{Broncode:algo1}. Tijdens de uitvoer van het algoritme zal achtereenvolgens elke noot in het muziekstuk overlopen worden. Telkens zal voor elk van de noten waar de beschouwde noot naartoe getransformeerd kan worden enkel de probabiliteit bijgehouden worden van het pad dat eindigt op deze noot en het meest waarschijnlijk is.

\subsubsection{Notatie}
Voor het beschrijven van het algoritme worden een aantal notaties en functies beschreven. Allereerst wordt volgende notatie ingevoerd:

\begin{framed}
\noindent
$\mathcal{T}:$ Set van alle transformaties die beschikbaar zijn voor het algoritme, in deze set zit ook telkens de nultransformatie die een noot nooit verandert.\\
$\mathcal{P}:$ Set van alle noten waar de vorige beschouwde noot naartoe getransformeerd kan worden.\\
$\mathcal{C}:$ Set van alle noten waar de huidige noot naartoe getransformeerd kan worden.\\
$AN$: Het aantal noten in het muziekstuk.\\
$x$: Noot uit de originele melodielijn die beschouwd wordt als huidige noot in het algoritme.
\end{framed}

De functie $transform$ geeft weer dat een vorige noot $p\in \mathcal{P}$ de huidige noot $x$ in de originele melodie onder transformatie $t\in \mathcal{T}$ afbeeldt naar noot $c\in \mathcal{C}$.

\begin{framed}
\noindent
$transform(p,x,t)=c$
\end{framed}

De functie $proxProb$ geeft het logaritme terug van de probabiliteit gegeven door de \textit{proximity} parameter van het RPK-model. Als voor de overgang van noot $p\in \mathcal{P}$ naar noot $c\in \mathcal{C}$, de afstandsprobabiliteit tussen deze twee noten gelijk is aan $d$, dan geeft de functie $proxProb$ de volgende waarde terug

\begin{framed}
\noindent
$proxProb(p,c)=log(d)$
\end{framed}

De functie $posProb$ geeft het logaritme terug van de probabiliteit van een noot die enkel afhangt van de toonhoogte van de noot zelf. Deze probabiliteiten komen overeen met de \textit{range} en de \textit{key} parameters uit het RPK-model. Deze probabiliteit is dus onafhankelijk van de voorgaande noot. stel dat voor een noot $c\in \mathcal{C}$ de probabiliteit door de \textit{range} parameter gegeven wordt door $r$. Stel ook dat de probabiliteit bepaald door de \textit{key} gegeven wordt door $k$. Nu wordt de totale $posProb$ gegeven door:

\begin{framed}
\noindent
$posProb(c)=log(r) + log(k)$
\end{framed}

\subsubsection{Maximale probabiliteit}
$prob(c)$ geeft het logaritme van de probabiliteit weer van het meest waarschijnlijke pad dat eindigt op deze noot $c\in \mathcal{C}$.

Wanneer de eerste noot van het muziekstuk beschouwd wordt en $x$ dus de waarde heeft van deze eerste noot, zal de probabiliteit van het pad dat eindigt op deze noot gelijk zijn aan 1. De probabiliteit van alle andere noten is gelijk aan 0. Dit komt omdat de eerst noot van een muziekstuk nooit getransformeerd wordt. Aangezien het algoritme werkt met de logaritmen van deze probabiliteiten zullen de waarden respectievelijk op 0 en $-\infty$ gezet worden. Wanneer $x$ gelijk is aan de eerste noot van het originele muziekstuk geldt dus:

\begin{framed}
\noindent
$\forall c\in \mathcal{C}: \begin{cases} 
prob(c)=0 &\mbox{if } c==x\\ 
prob(c)=-\infty &\mbox{if } c!=x \end{cases}$
\end{framed}

Voor alle mogelijke noten op andere posities van het muziekstuk geldt dat:

\begin{framed}
\noindent
\begin{multline}
\forall c\in \mathcal{C}: 
prob(c) = max(prob(p) + proxProb(p,c) + posProb(c)) \\
| \forall p\in \mathcal{P}: \exists t\in \mathcal{T}: transform(p,x,t)=c
\end{multline}
\end{framed}

Door deze functie achtereenvolgens toe te passen voor alle noten van het muziekstuk kan er bepaald worden wat de probabiliteit is van het beste pad, gebruik makend van de gegeven transformaties uit $\mathcal{T}$. Dit zal namelijk gelijk zijn aan de hoogste probabiliteit van alle noten $c\in \mathcal{C}$ in de laatste stap van het algoritme.

\subsubsection{Optimale pad}
We zijn natuurlijk niet enkel ge\"interesseerd in de probabiliteit van het optimale pad, maar ook dit pad zelf, want dit gaat de gevonden melodie weergeven. Daarom wordt er voor de elke combinatie van een positie in de melodie en de mogelijke noten die op die positie kunnen voorkomen, bijgehouden waarvan het optimale pad kwam dat uitkwam bij die exacte noot op die positie. Om deze paden voor te stellen wordt de functie $path$ gebruikt. Meer algemeen, stel we zijn op positie $i$ in het algoritme (we beschouwen dus de $i$-de noot van het muziekstuk), dan zal wanneer het optimale pad dat eindigt op noot $n$ komt van een zekere $m$, het volgende gelden:

\begin{framed}
\noindent
$path(i,n)=m$
\end{framed}

Wanneer het algoritme voor alle mogelijke noten op elke positie in de melodie de maximale probabiliteit bepaald heeft kan het optimale pad gevonden worden. Dit gebeurt zoals beschreven in algoritme \ref{alg:opt_pad}.

\begin{algorithm}
\caption{Optimaal pad}\label{alg:opt_pad}
\begin{algorithmic}
\State $note=argmax_{c \in \mathcal{C}}(prob(c))$
\State $Path.addToFront(note)$
\For {($i = AN: i\geq 2: i=i-1$)}{}	
	\State $note = path(i,note)$
	\State $Path.addToFront(note)$
\EndFor
\State \Return $Path$
\end{algorithmic}
\end{algorithm}

Intern in het algoritme worden al deze $path$ relaties opgeslagen in een $(n\times 14)$-\textit{matrix}. voor elk van de $n$ posities in de melodie worden voor alle 14 noten waar theoretisch naar getransformeerd kan worden, de index bijgehouden van de noot op de vorige positie waarvan het optimale pad kwam. Een voorbeeld van hoe hier een pad uit gereconstrueerd kan worden, is afgebeeld in figuur \ref{figuur:matrix}. Voor twee verschillende noten op positie $i$ in het muziekstuk wordt het optimale pad om tot die noot te geraken afgebeeld. De waarden die in de vakjes staan zijn telkens indexen die verwijzen naar een noot op de vorige positie.

\begin{figure}[!ht]
  \centering
  \includegraphics[width=0.75\textwidth]{4_Efficient_Toepassen_Transformatie/matrix}
  \caption{Illustratie van de reconstructie van het optimale pad voor 2 verschillende noten op positie $i$.}
  \label{figuur:matrix}
\end{figure}

\subsection{Performantie en geheugencomplexiteit}
De eerste parameter waarvan het algoritme afhankelijk is, is de lengte van de melodielijn of met andere woorden het aantal noten (AN) in de invoer. Ook het aantal beschikbare transformaties (AT) heeft een invloed (in het voorbeeld beschreven in appendix \ref{Broncode:algo1} wordt gebruik gemaakt van slechts 2 transformaties, maar het algoritme werkt voor eender welk aantal transformaties dat gedefinieerd wordt).

\subsubsection{Tijd} 
De snelheid van het algoritme is lineair afhankelijk van beide van deze parameters. Indien de lengte van het originele melodietje met een factor f omhoog gaat en de rest constant blijft dan gaat ook het aantal stappen in het algoritme met een factor f omhoog. Het rekenwerk per stap blijft echter gelijk. Wanneer het aantal transformaties met een factor t omhoog gaat en de rest constant blijft, dan zal het aantal stappen onveranderd blijven. Het rekenwerk per stap gaat wel met een factor t omhoog (op het constante rekenwerk per stap van het `niet transformeren' na).

\begin{center}
\underline{Tijd:} $\mathcal{O}(AN \times AT)$
\end{center}

\subsubsection{Geheugen}
De hoeveelheid geheugen die nodig is voor de uitvoer van het algoritme is lineair afhankelijk van de lengte van de originele melodie. Dit komt omdat de \textit{matrix} array als een van zijn dimensies deze lengte heeft. Wanneer de lengte van het originele stukje met een factor f omhoog gaat, zal de grootte van deze array dus ook met een factor f omhoog gaan. De grootte van de andere twee gebruikte arrays(\textit{past} en \textit{current}) is onveranderlijk ten opzichte van die lengte. De transformaties zelf moeten natuurlijk ook opgeslagen worden, waardoor de het geheugengebruik ook lineair afhankelijk is van het aantal transformaties. maar in het totale geheugengebruik is de \textit{matrix}-array dominant en opzichte van de opslag van de respresentaties van de transformaties, aangezien deze zo veel groter is. Het aantal gebruikte transformaties heeft dus weinig effect op het geheugengebruik van het algoritme. En in het algemeen kan er dus veronderstel worden dan het aantal gebruikte transformaties het geheugengebruik niet merkbaar be\"invloedt.

\begin{center}
\underline{Geheugen:} $\mathcal{O}(AN)$
\end{center}

\section{Beste sequentie met minimum transformatie lengte}
\label{ETT:algo2}

\subsection{Doelstelling}
Het algoritme dat in dit onderdeel beschreven wordt heeft dezelfde doelstelling als het algoritme beschreven in onderdeel \ref{ETT:algo1}. Enkel wordt dit algoritme aan nog een extra restrictie onderworpen.

Zo zal dit algoritme afhankelijk zijn van drie parameters. De eerste twee parameters zijn een originele melodielijn en een aantal toegelaten transformaties. Als extra parameter is dit algoritme nog afhankelijk van een opgegeven minimum transformatie lengte.

Dit betekent dat het algoritme enkel een deel van de originele melodielijn mag transformeren als het voor minstens dit opgegeven aantal opeenvolgende noten dezelfde transformatie uitvoert.

\subsection{Idee van het algoritme}
\subsubsection{Hoog niveau idee van het algoritme en notatie}
Ook nu zal er gebruik gemaakt worden van de principes van \textit{dynamic programming}. Dit zal enkel op een verschillende manier gebeuren dan bij het vorige algoritme, aangezien de opgegeven minimumlengte verhindert om eenzelfde data representatie te gebruiken. Ook zullen dezelfde regels gelden als in het overeenkomstige onderdeel uit \ref{ETT:algo1} wat het bereik van een tranformatie betreft.

Het idee van het algoritme bestaat erin om elke noot van het muziekstukje chronologisch te overlopen. Bij elke noot uit het originele stuk zijn er dan een aantal mogelijkheden om het beste pad te bepalen dat eindigt op een van de 14 mogelijke noten waarnaar de originele noot getransformeerd kan worden.
 
Als in het vervolg van de tekst gesproken wordt over een ``geldig pad'', dan wordt hier een pad mee bedoeld dat de regels van de minimumlengte voor transformatie respecteert.

\subsubsection{Verschillende mogelijkheden om tot een optimaal pad te komen} 
In dit onderdeel wordt een intu\"itieve beschrijving gegeven van hoe het optimale pad tot op een bepaalde positie in het muziekstuk kan berekend worden. Dit, gebruik makend van de optimale paden en bijhorende probabiliteiten van de noten op vorige posities in het muziekstuk.

Een eerste mogelijkheid voor een optimaal pad is de uitbreiding van eender welk optimaal pad dat eindigt op een van de noten waar de vorige noot uit het muziekstuk naar getransformeerd kan worden.
 
Een tweede mogelijkheid is het uitbreiden van een optimaal pad dat geldig is, eindigt op de vorige noot en eindigt met transformatie f uit te breiden met dezelfde transformatie f.
 
Tot slot is er ook nog de mogelijkheid om eender welk optimaal en geldig pad dat een lengte ML korter is dan het huidige uit te breiden met ML keer dezelfde transformatie. Op deze manier kunnen alle optimale paden bekomen worden die aan de vooropgestelde eisen voldoen.

\subsection{Werking van het algoritme}
In dit onderdeel zal de werking van het algoritme beschreven worden. Als extra referentie voor de lezer is de broncode bijgevoegd in appendix \ref{Broncode:algo2}. Tijdens de uitvoer van het algoritme zal achtereenvolgens elke noot in het muziekstuk overlopen worden. Nu zal voor elke transformatie apart bijgehouden worden wat de optimale paden en probabiliteiten zijn voor alle verschillende noten waar de huidige noot naartoe getransformeerd kan worden en die eindigen met die specifieke transformatie. 

\subsubsection{Notatie}
Voor het beschrijven van het algoritme worden een aantal notaties en functies beschreven. Allereerst wordt volgende notatie ingevoerd:

\begin{framed}
\noindent
$\mathcal{T}:$ Set van alle transformaties die beschikbaar zijn voor het algoritme, in deze set zit ook telkens de nultransformatie die een noot nooit verandert.\\
$nt$: De nultransformatie die ook in $\mathcal{T}$ zit.\\
$ML$: De minimum transformatie lengte.\\
$\mathcal{P_{ML}}:$ Set van alle noten waar de noot die $ML$ posities voor de huidige noot naartoe getransformeerd kan worden.\\
$\mathcal{P}:$ Set van alle noten waar de vorige beschouwde noot naartoe getransformeerd kan worden.\\
$\mathcal{C}:$ Set van alle noten waar de huidige noot naartoe getransformeerd kan worden.\\
$AN$: Het aantal noten in het muziekstuk.\\
$x$: Noot uit de originele melodielijn die beschouwd wordt als huidige noot in het algoritme.\\
$i$: De index van de noot die momenteel beschouwd wordt in het algoritme.\\
$OL$: Geordende lijst van de noten in de originele melodie.
\end{framed}

De functie $transform$ geeft weer dat een vorige noot $p\in \mathcal{P}$ de huidige noot $x$ in de originele melodie onder transformatie $t\in \mathcal{T}$ afbeeldt naar noot $c\in \mathcal{C}$.

\begin{framed}
\noindent
$transform(p,x,t)=c$
\end{framed}

De functie $proxProb$ geeft het logaritme terug van de probabiliteit gegeven door de \textit{proximity} parameter van het RPK-model. Als voor de overgang van noot $p\in \mathcal{P}$ naar noot $c\in \mathcal{C}$, de afstandsprobabiliteit tussen deze twee noten gelijk is aan $d$, dan geeft de functie $proxProb$ de volgende waarde terug

\begin{framed}
\noindent
$proxProb(p,c)=log(d)$
\end{framed}

De functie $posProb$ geeft het logaritme terug van de probabiliteit van een noot die enkel afhangt van de toonhoogte van de noot zelf. Deze probabiliteiten komen overeen met de \textit{range} en de \textit{key} parameters uit het RPK-model. Deze probabiliteit is dus onafhankelijk van de voorgaande noot. stel dat voor een noot $c\in \mathcal{C}$ de probabiliteit door de \textit{range} parameter gegeven wordt door $r$. Stel ook dat de probabiliteit bepaald door de \textit{key} gegeven wordt door $k$. Nu wordt de totale $posProb$ gegeven door:

\begin{framed}
\noindent
$posProb(c)=log(r) + log(k)$
\end{framed}

Tot slot is er ook nog een algoritme $extendPath$ dat berekent wat het logaritme van de probabiliteit is van de uitbreiding van een pad. Deze uitbreiding moet van lengte $l$ zijn en volgens een meegegeven transformatie $t\in \mathcal{T}$ bekomen worden. Dit algoritme krijgt ook het pad $path$ waarop deze uitbreiding moet uitgevoerd worden mee. Tot slot wordt aan dit algoritme wordt ook een lijst $list$ van opeenvolgende noten meegegeven, dit zijn de noten van het originele muziekstuk waarop deze transformatie toegepast moet worden. Dit algoritme geeft niet alleen het logaritme van de probabiliteit van deze uitbreiding mee, maar ook het volledige pad dat na uitbreiding bekomen wordt.

\begin{algorithm}
\caption{$extendPath(l,t,Path,list)$}\label{alg:opt_pad}
\begin{algorithmic}
\State $Prob = 0$
\State $p$ = $Path.last$
\For {($i = 1: i \leq l: i=i+1$)}{}	
	\State $x = list[i]$
	\State $c = transform(p,x,t)$
	\State $Path.add(c)$
	\State $Prob = Prob + proxProb(p,c) + posProb(c)$
	\State $p = c$
\EndFor
\State \Return $(Prob, Path)$ 
\end{algorithmic}
\end{algorithm}

\subsubsection{Optimale paden voor verschillende transformaties}
De reden dat paden die eindigen op verschillende transformaties apart bijgehouden worden heeft twee redenen. 

Ten eerste zal, wanneer het algoritme op voorhand weet dat een bepaald pad geldig is (voldoet aan de voorwaarden voor minimum transformatie lengte) en eindigt op een zeker transformatie $t$, weten dat dit pad enkel door dezelfde transformatie $t$ kan uitgebreid worden naar enkel de volgende noot. Voor andere transformaties zal minstens $ML$ keer die transformatie moeten toegepast worden. Met deze voorstelling is het meteen duidelijk welke transformatie ook voor een noot mag uitgevoerd worden en welke voor minstens $ML$ volgende noten moet uitgevoerd worden om terug een geldig pad te bekomen.

Ten tweede heeft het ook te maken met de extra eis van de minimum transformatie lengte. Indien alle optimale paden zouden bijgehouden worden met een $path$ relatie zoals in het vorige algoritme, die onafhankelijk zou zijn van de transformatie waarop ge\"eindigd werd, dan kan tijdens de uitvoer van het algoritme de regel van de minimum transformatie lengte geschonden worden. Een voorbeeld hiervan wordt hieronder gegeven:

\begin{framed}
\noindent
Stel dat:
$ML=2$\\
$path(3,W_1) = V_1$\\
$path(2,V_1) = U_1$\\
$path(3,W_2) = V_2$\\
$path(2,V_2) = U_2$\\
Dan zijn momenteel de optimale paden voor $W_1$ en $W_2$, respectievelijk $U_1-V_1-W_1$ en $U_2-V_2-W_2$.\\
Stel verder dat het optimale pad voor $W_1$ bekomen werd door tweemaal een transformatie $t_1 \in \mathcal{T}$ toe te passen en dat het optimale pad voor $W_2$ bekomen werd door tweemaal $t_2 \in \mathcal{T}$ toe te passen.\\
Beide paden zijn momenteel dus geldig onder de voorwaarde van de minimum transformatie lengte.\\
Stel nu dat er in het verdere verloop van het algoritme nog een beter pad gevonden wordt dat eindigt in $W_2$ en dat verloopt volgens $U_2-V_1-W_2$. Stel ook dat dit pad gevonden werd door het tweemaal toepassen van $t_3 \in \mathcal{T}$. Nu wordt de $path$ functie aangepast zodat deze de juiste waarden teruggeeft.\\
Nu geldt:\\
$path(3,W_1) = V_1$\\
$path(2,V_1) = U_2$\\
$path(3,W_2) = V_1$\\
$path(2,V_2) = U_2$\\
Dan zijn nu de optimale paden voor $W_1$ en $W_2$, respectievelijk $U_2-V_1-W_1$ en $U_2-V_1-W_2$.\\
Het optimale pad van $W_1$ is nu geen geldig optimaal pad meer aangezien het door het achtereenvolgens toepassen van telkens een maal de transformatie $t3$ en $t1$ niet meer voldoet aan de voorwaarden van de minimum transformatie lengte.
\end{framed}

\subsubsection{Maximale probabiliteit}
$prob(c,t)$ geeft het logaritme van de probabiliteit weer van het meest waarschijnlijke pad dat eindigt op deze noot $c\in \mathcal{C}$. Deze noot (en zelfs minstens de laatste $ML$ noten) moet in dit geval bekomen zijn door gebruik van transformatie $t \in \mathcal{T}$.

Wanneer de eerste noot van het muziekstuk beschouwd wordt en $x$ dus de waarde heeft van deze eerste noot, zal de probabiliteit van het pad dat eindigt op deze noot gelijk zijn aan 1. De probabiliteit van alle andere noten is gelijk aan 0. Dit komt omdat de eerste noot van een muziekstuk nooit getransformeerd wordt. Dit zal enkel het geval zijn voor de nultransformatie. Voor alle andere transformaties zal eender welke noot een kans 0 hebben om deel uit te maken van het optimale pad aangezien er geen enkel optimaal pad kan zijn dat eindigt op een transformatie als dat pad maar een noot kan bevatten. Aangezien het algoritme werkt met de logaritmen van deze probabiliteiten zullen de waarden respectievelijk op 0 en $-\infty$ gezet worden. Wanneer $x$ gelijk is aan de eerste noot van het originele muziekstuk en transformatie $t\in \mathcal{T}$ beschouwd wordt geldt dan:

\begin{framed}
\noindent
$\forall c\in \mathcal{C}, t\in \mathcal{T} \begin{cases} 
prob(c,t)=0 &\mbox{if } c==x $ \& $ t=nt\\ 
prob(c,t)=-\infty &\mbox{if } c==x $ \& $ t!=nt\\ 
prob(c,t)=-\infty &\mbox{if } c!=x \end{cases}$
\end{framed}

Voor alle andere noten en transformaties op alle andere posities in het muziekstuk geldt:

\begin{framed}
\noindent
\begin{multline}
\forall t\in \mathcal{T}: 
ProbSame(c,t) = max(prob(p,t) \\ 
+ extendPath(1,t,path(i-1,p,t),OL[i]).prob) \\
| \forall p\in \mathcal{P}: transform(p,x,t)=c
\end{multline}

\begin{multline}
\forall t\in \mathcal{T}: 
ProbDifferent(c,t) = max(prob(p_{ML},t) \\ 
+ extendPath(i-ML,t,path(i-ML,p_{ML},t1).prob,OL[(i-ML+1)..i]) \\
| \forall p_{ML}\in \mathcal{P_{ML}}, \forall t1\in \mathcal{T}:\\
t1!=t, extendPath(i-ML,t,path(i-ML,p_{ML},t1).path.last=c
\end{multline}

\begin{multline}
\forall c\in \mathcal{C}, t\in \mathcal{T}: 
prob(c,t) = max(ProbSame(c,t), ProbDifferent(c,t))
\end{multline}
\end{framed}

Door deze functie achtereenvolgens toe te passen voor alle noten van het muziekstuk kan er bepaald worden wat de probabiliteit is van het beste pad, gebruik makend van de gegeven transformaties uit $\mathcal{T}$ rekening houdend met de minimum transformatie lengte $ML$. Dit zal namelijk gelijk zijn aan de hoogste probabiliteit van de combinatie van alle noten $c\in \mathcal{C}$ en transformaties $t\in \mathcal{T}$ in de laatste stap van het algoritme.

\subsubsection{Optimale pad}
$path(i,n,t)$ stelt het volledige pad voor dat optimaal is onder de gegeven voorwaarden en dat eindigt bij noot $n$ op positie $i$ en door te eindigen met transformatie $t\in \mathcal{T}$. Initieel zijn al deze paden volledig leeg, enkel het pad dat eindigt op de nultransformatie en op de noot die gelijk is aan de startnoot bestaat uit exact deze noot. Telkens er een pad gevonden wordt dat eindigt op  noot $n$ en met transformatie $t\in \mathcal{T}$, dan zal niet enkel de $prob(n,t)$ waarde aangepast worden, maar ook $path(i,n,t)$ zal het nieuwe beste pad bevatten voor de combinatie van deze parameters.

Aangezien volledige beste paden bijgehouden worden moet er geen werk verricht worden om het beste pad te bepalen. Er moet enkel dat pad gekozen worden van lengte $AN$ dat de hoogste probabiliteit oplevert. Voor de combinatie van alle noten $c\in \mathcal{C}$ en transformaties $t\in \mathcal{T}$ is er zo een optimaal pad dat hier een kandidaat voor is. Van al deze kandidaten zal het pad met de hoogste probabiliteit het optimale pad zijn voor het geheel.

\subsection{Performantie en geheugencomplexiteit}
In dit algoritme zijn er drie parameters waar rekening mee dient gehouden te worden bij het bespreken van de tijds- en geheugencomplexiteit. Deze drie parameters zijn het aantal noten in de originele melodielijn (AN), het aantal beschikbare transformaties (AT) en de minimum transformatie lengte (ML).

\subsubsection{Tijd}
De performantie van het algoritme is lineair afhankelijk van het aantal noten. Dit komt doordat voor elke noot in het oorspronkelijk muziekstuk een stap in het algoritme moet uitgevoerd worden. Het rekenwerk per stap verandert niet wanneer de lengte van het stuk verandert.\\ 
De snelheid van het algoritme is kwadratisch afhankelijk van het aantal toegestane transformaties. Dit komt omdat voor elke stap in het algoritme we voor elke transformatie gaan proberen een optimaal pad te vinden dat kan vertrekken van bij tussenoplossingen horende bij alle andere transformaties. Het aantal stappen in het algoritme verandert niet wanneer het aantal transformaties verandert. In totaal geeft dit dus een kwadratische afhankelijkheid.\\ 
Tot slot is de snelheid van het algoritme ook afhankelijk van de minimum transformatie lengte volgens $\mathcal{O}(ML \times (AN-ML))$. zowel het aantal stappen in het algoritme als het aantal berekende paden per stap zal onveranderd blijven wanneer deze parameter verandert. Het extra rekenwerk dat gecre\"eerd wordt door het moeten uitbreiden van paden die ML korter zijn dan de huidige lengte is lineair afhankelijk van de grootte van ML. Maar het aantal paden dat zo berekend moet worden is lineair afhankelijk van (L-ML). Hierdoor zal de totale rekentijd afhankelijk zijn van het product van deze twee.

\begin{center}
\underline{Tijd:} $\mathcal{O}(AN \times ML \times (AN-ML) \times AT^2)$
\end{center}

\subsubsection{Geheugen}
De totale opslagcapaciteit die noodzakelijk is voor de uitvoer van het algoritme is ten eerste lineair afhankelijk van de lengte van de invoersequentie van noten. Voor alle mogelijke transformaties zullen namelijk optimale paden bijgehouden worden en de lengte van deze paden is altijd van dezelfde grootte-orde als de lengte van het originele.\\ 
Ten tweede is het geheugengebruik ook lineair afhankelijk van de minimum transformatie lengte. Er worden namelijk optimale sub paden bijgehouden in het algoritme voor de laatste ML beschouwde noten. De lengte van deze paden is ook elk van dezelfde grootte-orde.\\ 
Tot slot is er nog het aantal transformaties. Ook deze parameter zal een lineaire invloed hebben op het geheugengebruik. Dit aangezien voor elke beschikbare transformatie even veel optimale paden bijgehouden worden die eindigen op deze transformatie.  

\begin{center}
\underline{Geheugen:} $\mathcal{O}(AN \times ML \times AT)$
\end{center}

%%% Local Variables: 
%%% mode: latex
%%% TeX-master: "masterproef"
%%% End: 

\chapter{Experimenten en Resultaten}
\label{hoofdstuk:ER}
In dit hoofdstuk worden de belangrijkste experimenten besproken die uitgevoerd in het verloop van deze masterproef. Deze hebben zowel betrekking op de besproken melodische transformaties, de algoritmes om deze transformaties te combineren en ook het RPK-model dat gebruikt werd om deze transformaties te evalueren.\\
De resultaten worden meestal weergegeven op een plot waarbij op de y as de gemiddelde probabiliteit staat. Deze probabiliteit staat voor de gemiddelde probabiliteit van voorkomen van een noot (gegeven de vorige noot) volgens het RPK-model in alle muziekstukken waarop er getest werd. Als er dus in dit hoofdstuk gesproken wordt over een gemiddelde probabiliteit van een bepaald muziekstuk dan betekent dit de gemiddelde probabiliteit van een noot in dit muziekstuk.

\section{Transformaties combineren: 1 transformatie, meerdere iteraties}
\label{experiment:1}
\subsection{Beschrijving experiment}
Dit experiment heeft betrekking tot het algoritme dat besproken werd in onderdeel \ref{ETT:algo1}. Dit experiment gaat nagaan wat de invloed is van het aantal iteraties van het aantal iteraties van dit algoritme op de consonantiescore van het totale muziekstuk. Er wordt in dit algoritme slechts gebruik gemaakt van een transformatie. Deze gebruikte transformatie wordt weergegeven in tabel \ref{tabel:exp1}.

\begin{table}
  \centering
  \begin{tabular}{c | c c c c c c c c }
    Diff (mod 8) & 0 & 1 & 2 & 3 & 4 & 5 & 6 & 7 \\
    \hline
    \hline
    Verhoging & 5 & -4 & 1 & -3 & 1 & 1 & 2 & 3 \\
  \end{tabular}
  \caption{Transformatie gebruikt in het experiment van onderdeel \ref{experiment:1}.}
  \label{tabel:exp1}
\end{table}

Deze test wordt uitgevoerd op 100 muziekstukken uit het Essencorpus. De gemiddelde probabiliteit van de originele stukken wordt berekend alsook de gemiddelde probabiliteit van het muziekstuk dat optimaal is volgens het RPK-model in de toonaard van de 100 stukken. Nu kan er gekeken worden naar hoe snel de consonantiescore zich gaat verplaatsten van die van het originele naar die van de theoretisch best mogelijke volgens het model afhankelijk van het aantal iteraties dat het algoritme wordt uitgevoerd.

\subsection{Resultaten}

\begin{figure}[!ht]
  \centering
  \includegraphics[width=0.75\textwidth]{5_Experimenten_Resultaten/exp1_res}
  \caption{Resultaten van het experiment uitgevoerd in deel \ref{experiment:1}. De groene lijn staat voor de gemiddelde probabiliteit van een noot in de originele melodie, de rode lijn voor de gemiddelde probabiliteit van de theoritisch beste melodielijn in de toonaarden waarop getest werd en de blauwe lijn geeft de gemiddelde probabiliteit van een noot weer na uitvoer van het algoritme na een verschillend aantal iteraties.}
  \label{figuur:exp1}
\end{figure}

In figuur \ref{figuur:exp1} worden de restultaten weergegeven van dit experiment. De groene lijn op de figuur geeft de gemiddelde probabiliteit weer van alle melodie\"en waarop getest werd. De rode lijn geeft voor al deze melodie\"en de gemiddelde waarde mee voor het theoretisch beste muziekstuk dat volgens het RPK-model gemaakt kan worden in deze toonaard. Tot slot geeft de blauwe lijn de gemiddelde probabiliteit weer van een noot in de getransformeerde melodie, na toepassing van 1 tot 10 iteraties van de transformatie.\\
Er valt duidelijk op dat de eerste paar iteraties nog een redelijke verhoging van de probabiliteit teweeg brengt, maar dat na een vijftal iteraties gemiddeld gezien een soort van maximum bereikt is dat met deze transformatie kan bekomen worden. De reden dat deze waarde nog zo ver onder het theoretische maximum ligt, heeft er vooral mee te maken dat er maar 1 mogelijke transformatie is dat het algoritme mag gebruiken, hierdoor zijn er nog steeds maar een zeer beperkt aantal mogelijkheiden om een bepaalde noot te transformeren, en kunnen bijgevolg nog steeds de meeste noten niet bereikt worden vanuit eender welke noot.

\section{Transformaties combineren: meerdere transformaties, 1 iteratie}
\label{experiment:2}
\subsection{Beschrijving experiment}
Dit experiment heeft betrekking tot het algoritme dat besproken werd in onderdeel \ref{ETT:algo1}. Het experiment gaat nagaan wat de invloed is van het aantal verschillende toegelaten transformaties op de consonantiescore van het totale muziekstuk. Zo zijn de vijf transformaties die gebruikt zullen worden weergegeven in tabel \ref{tabel:exp2}. Er wordt nu telkens slechts een iteratie van het algoritme uitgevoerd.\\

\begin{table}
  \centering
  \begin{tabular}{c | c c c c c c c c }
    Diff (mod 8) & 0 & 1 & 2 & 3 & 4 & 5 & 6 & 7 \\
    \hline
    \hline
    Verhoging transformatie 1 & 5 & -4 & 1 & -3 & 1 & 1 & 2 & 3 \\
    \hline
    Verhoging transformatie 2 & 1 & 3 & 4 & -5 & -1 & 6 & 5 & -1 \\
    \hline
    Verhoging transformatie 3 & 1 & 4 & 5 & -3 & 2 & -1 & 1 & 0 \\
    \hline
    Verhoging transformatie 2 & 4 & 6 & -2 & 4 & 2 & 6 & -4 & 2 \\
    \hline
    Verhoging transformatie -2 & -3 & -2 & 3 & -1 & 4 & -3 & 2 & -4 \\
  \end{tabular}
  \caption{Transformaties gebruikt in het experiment van onderdeel \ref{experiment:2}.}
  \label{tabel:exp2}
\end{table}

Deze test wordt uitgevoerd op 50 muziekstukken uit het Essencorpus. De gemiddelde probabiliteit van de originele stukken wordt berekend alsook de gemiddelde probabiliteit van het muziekstuk dat optimaal is volgens het RPK-model in de toonaard van de 50 stukken. Nu kan er weer gekeken worden naar hoe snel de consonantiescore zich gaat verplaatsten van die van het originele naar die van de theoretisch best mogelijke volgens het model afhankelijk van het aantal transformaties dat aan het algoritme ter beschikking wordt gesteld. Het experiment wordt eerst uitgevoerd met slechts een mogelijke transformatie. Dit zal dan transformatie 1 uit de tabel zijn. Daarna wordt het experiment uitgevoerd met 2 mogelijke transformaties. Dit zullen dan de eerste twee transformaties uit de tabel zijn enz..

\subsection{Resultaten}

\begin{figure}[!ht]
  \centering
  \includegraphics[width=0.75\textwidth]{5_Experimenten_Resultaten/exp2_res}
  \caption{Resultaten van het experiment uitgevoerd in deel \ref{experiment:2}. De groene lijn staat voor de gemiddelde probabiliteit van een noot in de originele melodie, de rode lijn voor de gemiddelde probabiliteit van de theoritisch beste melodielijn in de toonaarden waarop getest werd en de blauwe lijn geeft de gemiddelde probabiliteit van een noot weer na uitvoer van het algoritme afhankelijk van het aantal transformaties dat ter beschikking gesteld was aan het algoritme.}
  \label{figuur:exp2}
\end{figure}

In figuur \ref{figuur:exp2} worden de restultaten weergegeven van dit experiment. Het is duidelijk dat een  hoger aantal transformaties ook telkens een betere score teruggeeft. Als we de resultaten van dit experiment vergelijken met dat uit onderdeel \ref{experiment:1}, dan merken we op dat het aantal transformaties een grotere impact heeft op de probabiliteit dan het aantal iteraties. De grootste reden hiervoor is dat een extra transformaties er voor kan zorgen dat er voor elke een noot in het muziekstuk een extra mogelijke noot is waarnaar hij getransformeerd kan worden. Dit zorgt voor enorm veel extra mogelijkheden waardoor er hogere probabiliteiten kunnen bekomen worden dan in het geval waarbij het aantal iteraties verhoogd wordt in plaats van het aantal mogelijke transformaties.  


\section{Transformaties combineren: minimum transformatie lengte}
\label{experiment:3}
\subsection{Beschrijving experiment}
Dit experiment heeft betrekking tot het algoritme dat besproken werd in onderdeel \ref{ETT:algo2}. Het experiment gaat nagaan wat de invloed is van de minimum transformatielengte op de consonantiescore van het totale muziekstuk. Er wordt in dit experiment telkens gebruik gemaakt van 2 transformaties die beschreven zijn in tabel \ref{tabel:exp3}. Er wordt telkens slechts een iteratie van het algoritme uitgevoerd.\\

\begin{table}
  \centering
  \begin{tabular}{c | c c c c c c c c }
    Diff (mod 8) & 0 & 1 & 2 & 3 & 4 & 5 & 6 & 7 \\
    \hline
    \hline
    Verhoging transformatie 1 & 5 & -4 & 1 & -3 & 1 & 1 & 2 & 3 \\
    \hline
    Verhoging transformatie 2 & 1 & 3 & 4 & -5 & -1 & 6 & 5 & -1 \\
  \end{tabular}
  \caption{Transformaties gebruikt in het experiment van onderdeel \ref{experiment:3}.}
  \label{tabel:exp3}
\end{table}

Dit experiment wordt uitgevoerd op 50 muziekstukken uit het Essencorpus en dit voor alle waarden van de minimum transformatie lengte tussen 1 en 10. Voor elk van deze 10 gevallen wordt de gemiddelde probabiliteit van het getransformeerde muziekstuk berekend.

\subsection{Resultaten}

\begin{figure}[!ht]
  \centering
  \includegraphics[width=0.75\textwidth]{5_Experimenten_Resultaten/exp3_res}
  \caption{Resultaten van het experiment uitgevoerd in deel \ref{experiment:3}. De groene lijn staat voor de gemiddelde probabiliteit van een noot in de originele melodie, de rode lijn voor de gemiddelde probabiliteit van de theoritisch beste melodielijn in de toonaarden waarop getest werd en de blauwe lijn geeft de gemiddelde probabiliteit van een noot weer na uitvoer van het algoritme afhankelijk van de minimum transformatie lengte.}
  \label{figuur:exp3}
\end{figure}

Figuur \ref{figuur:exp3} geeft de resultaten weer van het experiment. Het is duidelijk dat een hogere minimum transformatie lengte leidt tot een kleinere verbetering van de consonantiescore, wat te verwachten was. De grafiek is ook monotoon dalen voor stijgende waarde van de minimum transformatie lengte. Dit moet ook zo zijn aangezien alle transformaties die geldig zijn voor een zekere transformatie lengte ook altijd geldig zijn voor alle kortere transformatie lengtes.

\section{Transformaties combineren: Gelijkheid algoritmen voor transformatie lengte 1}
\label{experiment:4}
\subsection{Beschrijving experiment}
Dit experiment is opgezet als extra test om het geloof in de juiste werking van de algoritmes beschreven in \ref{ETT:algo1} en \ref{ETT:algo2} te versterken. Aangezien de implementaties van deze twee algoritmen toch op een aantal vlakken (vooral de voorstelling van de paden) verschillen van elkaar, is het interessant om voor een minimum transformatie lengte van 1 eens te kijken of de twee algoritmes hetzelfde resultaat geven. Dit zou normaal gezien altijd het geval moeten zijn aangezien de twee algoritmes hetzelfde doel en dezelfde middelen hebben in het geval van een minimum transformatie lengte van 1. De transformatie waarvan gebruik gaat worden gemaakt staat beschreven in tabel \ref{tabel:exp4}.

\begin{table}
  \centering
  \begin{tabular}{c | c c c c c c c c }
    Diff (mod 8) & 0 & 1 & 2 & 3 & 4 & 5 & 6 & 7 \\
    \hline
    \hline
    Verhoging & 5 & -4 & 1 & -3 & 1 & 1 & 2 & 3 \\
  \end{tabular}
  \caption{Transformatie gebruikt in het experiment van onderdeel \ref{experiment:4}.}
  \label{tabel:exp4}
\end{table}

Voor deze transformatie gaat het experiment zoals het beschreven staat in onderdel \ref{experiment:1} herhaald worden maar dan ook voor het tweede algoritme. Dus voor een aantal iteraties van 1 tot en met 10 van het algoritme gaan de probabiliteiten die beide algoritmes opleveren voor dezelfde transformatie op dezelfde 100 testgevallen uit het Essencorpus vergeleken worden.

\subsection{Resultaten}

\begin{table}
  \centering
  \begin{tabular}{c | c | c }    
    \# iteraties & Algoritme 1 & Algoritme 2 \\
    \hline
    1 & -1.69 & -1.69\\
    2 & -1.57 & -1.57\\
    3 & -1.54 & -1.54\\
    4 & -1.51 & -1.51\\
    5 & -1.50 & -1.50\\
    6 & -1.50 & -1.50\\
    7 & -1.50 & -1.50\\
    8 & -1.50 & -1.50\\
    9 & -1.50 & -1.50\\
    10 & -1.50 & -1.50\\
  \end{tabular}
  \caption{Resulaten van experiment \ref{experiment:4}. Gemiddelde consonantiescores voor de twee algoritmen (logaritme van de probabiliteit) na een gegeven aantal iteraties. Algoritme 1 staat beschreven in \ref{ETT:algo1}, algoritme 2 is hetgene dat beschreven staat in \ref{ETT:algo2}.}
  \label{tabel:res4}
\end{table}

In tabel \ref{tabel:res4} staan de resultaten van dit experiment weergegeven. Beide algoritmen leveren dezelfde gemiddelde consonantiescore (logaritme van de gemiddelde probabiliteit) voor een muziekstukje na eenzelfde aantal iteraties gebruik makende van dezelfde transformatie. Dit versterkt de stelling dat de twee algoritmes wel degelijk werken zoals gewenst.

\section{Vergelijking van de twee melodische transformaties}
\label{experiment:5}
\subsection{Beschrijving experiment}

\subsection{Resultaten}

\begin{table}
  \centering
  \begin{tabular}{c | c c }    
    Transformatie & Consonantiescore & Probabiliteit(\%)\\
    \hline
    Oiginele melodie & -2.36 & 9.42\\
    Transformatie volgens \ref{MT:positie} & -3.72 & 2.43\\
    Transformatie volgens \ref{MT:afstand_vorige} & -3.19 & 4.12\\
  \end{tabular}
  \caption{Resulaten van experiment \ref{experiment:5}. Gemiddelde consonantiescores voor de twee soorten transformaties en de originele melodielijnen die getranformeerd werden. De twee geteste soorten van transformaties staan beschreven in onderdelen \ref{MT:positie} en \ref{MT:afstand_vorige}.}
  \label{tabel:res5}
\end{table}

\section{Vergelijking Fibonacci transformatie met gemiddelde transformatie}
\label{experiment:6}
\subsection{Beschrijving experiment}

\subsection{Resultaten}

\begin{table}
  \centering
  \begin{tabular}{c | c c }    
    Transformatie & Consonantiescore & Probabiliteit(\%)\\
    \hline
    Oiginele melodie & -2.36 & 9.42\\
    Transformatie volgens \ref{MT:positie} & -2.92 & 5.42\\
    Transformatie volgens \ref{MT:afstand_vorige} & -2.63 & 7.21\\
  \end{tabular}
  \caption{Resulaten van experiment \ref{experiment:6}. Gemiddelde consonantiescores voor de twee Fibonacci transformaties en de originele melodielijnen die getranformeerd werden.}
  \label{tabel:res6}
\end{table}

\section{Test Performantie algoritmes}
\label{experiment:7}
\subsection{Beschrijving experiment}

\subsection{Resultaten}

%%% Local Variables: 
%%% mode: latex
%%% TeX-master: "masterproef"
%%% End: 

\chapter{Besluit}
\label{hoofdstuk:B}

\section{Samenvatting}
In deze masterproef zijn allereerst twee methoden besproken voor de evaluatie van muziekstukken. Het gebruik van een neuraal netwerk voor polyfone muziekstukken had zijn tekortkomingen door het negeren van de context van een noot. Dit probleem werd opgelost door het RPK-model dat geschikt is voor monofone muziekstukken. Dit RPK-model kon daarna gebruikt worden voor de melodische evaluatie van verschillende transformaties.

Er werden twee melodische transformaties besproken. Een van deze transformaties diende vooral als \textit{baseline} en was zeer elementair en eenvoudig te implementeren. De andere transformatie dat zich baseert op de intervallen tussen opeenvolgende noten leverde betere resultaten (afwijking consonantiescore van de originele melodie ten opzichte van afwijking met een willekeurige transformatie is significant kleiner). De resultaten zijn, alhoewel ze duidelijk beter zijn dan in het willekeurige geval, wel nog steeds niet goed genoeg om te kunnen spreken van een ``goede'' transformatie. Dit komt omdat de bekomen consonantiescore van een stuk na deze transformatie gemiddeld gezien nog steeds significant minder goed is ten opzichte van die van het origineel.

Tot slot zijn er nog twee algoritmen ontworpen die zorgen voor het effici\"ent combineren van verschillende transformaties. Het eerste van deze algoritmes kan gegeven een aantal transformaties en een originele melodielijn, deze melodielijn transformeren naar een nieuwe melodielijn met een zo hoog mogelijke consonantiescore. Het tweede algoritme verwezenlijkt hetzelfde doel, maar heeft als extra voorwaarde dat een transformatie, wanneer deze wordt uitgevoerd, meteen voor een minimum aantal opeenvolgende noten uitgevoerd moet worden. Van deze algoritmen werd de afhankelijkheid van de verschillende parameters getest en in kaart gebracht. Van het tweede algoritme werd ook de berekende tijdscomplexiteit geverifi\"eerd. Deze algoritmen kunnen dus gebruikt worden om zo effici\"ent mogelijk (volgens het RPK-model) verschillende transformaties te combineren. Hierbij dient wel in het achterhoofd gehouden te worden dat het in deze algoritmen de bedoeling is de consonantiescore zo hoog mogelijk te krijgen. Dit leidt vaak niet tot interessante muziekstukken. Het juist kiezen van de parameters waarvan deze algoritmen afhankelijk zijn kan er wel voor zorgen dat de score gemiddeld gezien sneller of trager afwijkt van die van het origineel.

\section{Verder onderzoek}
In dit onderdeel worden nog enkele onderwerpen aangekaart die onderwerp zouden kunnen zijn voor verder onderzoek in dit onderzoeksdomein.

\subsection{Uitbreiding context RPK-model}
Het RPK-model brengt momenteel voor het berekenen van de probabiliteit van een muziekstuk voor elke noot, de voorafgaande noot in rekening. Het kan interessant zijn dit uit te breiden naar meerdere vorige noten aangezien deze, zij het in mindere mate, ook een invloed hebben op de waarschijnlijkheid van de huidige noot.\\
Momenteel wordt er ook enkel melodisch ge\"evalueerd, het ritme van het muziekstuk wordt namelijk genegeerd. Het zou ook interessant kunnen zijn, zelfs voor het louter evalueren van melodische transformaties, om ook het ritme in rekening te brengen. Men zou bijvoorbeeld kunnen verwachten dat hoe sneller de noten elkaar opvolgen hoe kleiner de gemiddelde sprongen in toonhoogte zullen zijn. Hierdoor zou de \textit{proximity}-component mede afhankelijk van de ritmische afstand kunnen gemaakt worden in plaat van enkel van de melodische afstand.

\subsection{Algoritme dat combineert met als doel het behouden van de consonantiescore}
De algoritmen die ontworpen zijn in dit onderzoek hebben als doel om de consonantiescore van een melodielijn zo hoog mogelijk te krijgen door het combineren van verschillende transformaties. Dit levert echter niet altijd interessante resultaten op in de praktijk. In het onderzoek is gesteld dat een transformatie als `goed' beschouwd wordt wanneer deze de consonantiescore zou weining mogelijk zou veranderen. In dit opzicht is het interessant om een algoritme te maken dat als doel zou hebben om een gegeven melodielijn te transformeren, gebruik makende van de beschikbare transformaties, naar een nieuwe melodielijn wiens consonantiescore die van het origineel zo dicht mogelijk benadert. Als extra voorwaarde zou dan kunnen opgelegd worden dat het nieuwe muziekstukje op melodisch vlak zo veel mogelijk zou moeten afwijken van het origineel zodat dit niet meer herkenbaar zou zijn na transformatie.

\subsection{Combinatie ritmische en melodische transformaties}
Deze masterproef beschrijft enkel melodische transformaties. Het kan echter interessant zijn om in een verder stadium van het onderzoek van muzikale transformaties te gaan kijken naar combinaties van melodische en ritmische transformaties. Zo zal een melodische transformatie bijvoorbeeld ook minder van het originele muziekstuk moeten aanpassen opdat het geheel niet aan het origineel zou doen denken. Dit omdat er ook nog een ritmische aanpassing zou zijn.

%%% Local Variables: 
%%% mode: latex
%%% TeX-master: "masterproef"
%%% End: 


% Indien er bijlagen zijn:
\appendixpage*          % indien gewenst
\appendix
\chapter{Broncode}
\label{app:broncode}

\section{Neuraal netwerk trainer}
\label{Broncode:ANN}
\lstinputlisting[language = Java, frame=single]{Appendix_0_Broncode/NeuralNetworkTrainer.java}

\section{RPK-model}
\label{Broncode:RPK}
\lstinputlisting[language = Python, frame=single]{Appendix_0_Broncode/RPK.py}

\section{Beste sequentie}
\label{Broncode:algo1}
\lstinputlisting[language = Python, frame=single]{Appendix_0_Broncode/BestSequenceCleaned.py}

\section{Beste sequentie met minimum transformatie lengte}
\label{Broncode:algo2}
\lstinputlisting[language = Python, frame=single]{Appendix_0_Broncode/BestCoordinatedSequenceCleaned.py}

%%% Local Variables: 
%%% mode: latex
%%% TeX-master: "masterproef"
%%% End: 

\chapter{IEEE\_Paper}
\label{app:paper}
In de bijlagen vindt men de data terug die nuttig kunnen zijn voor de
lezer, maar die niet essentieel zijn om het betoog in de normale tekst te
kunnen volgen. Voorbeelden hiervan zijn bronbestanden,
configuratie-informatie, langdradige wiskundige afleidingen, enz.

In een bijlage kunnen natuurlijk ook verdere onderverdelingen voorkomen,
evenals figuren en referenties\cite{h2g2}.

%%% Local Variables: 
%%% mode: latex
%%% TeX-master: "masterproef"
%%% End: 

\chapter{Poster}
\label{app:poster}

\includepdf[pages={1}]{Appendix_2_Poster/Poster_EliasMoons.pdf}

%%% Local Variables: 
%%% mode: latex
%%% TeX-master: "masterproef"
%%% End: 



\backmatter
% Na de bijlagen plaatst men nog de bibliografie.
% Je kan de  standaard "abbrv" bibliografiestijl vervangen door een andere.
\bibliographystyle{abbrv}
\bibliography{referenties}

\end{document}

%%% Local Variables: 
%%% mode: latex
%%% TeX-master: t
%%% End: 
