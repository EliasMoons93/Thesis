\chapter{Experimenten en Resultaten}
\label{hoofdstuk:ER}
In dit hoofdstuk worden de belangrijkste experimenten besproken die uitgevoerd in het verloop van deze masterproef. Deze hebben zowel betrekking op de besproken melodische transformaties, de algoritmes om deze transformaties te combineren en ook het RPK-model dat gebruikt werd om deze transformaties te evalueren.\\
De resultaten worden meestal weergegeven op een plot waarbij op de y as de gemiddelde probabiliteit staat. Deze probabiliteit staat voor de gemiddelde probabiliteit van voorkomen van een noot (gegeven de vorige noot) volgens het RPK-model in alle muziekstukken waarop er getest werd.

\section{Transformaties combineren: 1 transformatie, meerdere iteraties}
\label{experiment:1}
\subsection{Beschrijving experiment}
Dit experiment heeft betrekking tot het algoritme dat besproken werd in onderdeel \ref{ETT:algo1}. Dit experiment gaat nagaan wat de invloed is van het aantal iteraties van het aantal iteraties van dit algoritme op de consonantiescore van het totale muziekstuk. Er wordt in dit algoritme slechts gebruik gemaakt van een transformatie. Deze gebruikte transformatie wordt weergegeven in tabel \ref{tabel:exp1}.

\begin{table}
  \centering
  \begin{tabular}{c | c c c c c c c c }
    Diff (mod 8) & 0 & 1 & 2 & 3 & 4 & 5 & 6 & 7 \\
    \hline
    \hline
    Verhoging & 5 & -4 & 1 & -3 & 1 & 1 & 2 & 3 \\
  \end{tabular}
  \caption{Transformatie gebruikt in het experiment van onderdeel \ref{experiment:1}.}
  \label{tabel:exp1}
\end{table}

Deze test wordt uitgevoerd op 100 muziekstukken uit het Essencorpus. De gemiddelde consonantiescore van de originele stukken wordt berekend alsook de gemiddelde consonantiescore van het muziekstuk dat optimaal is volgens het RPK-model in de toonaard van de 100 stukken. Nu kan er gekeken worden naar hoe snel de consonantiescore zich gaat verplaatsten van die van het originele naar die van de theoretisch best mogelijke volgens het model afhankelijk van het aantal iteraties dat het algoritme wordt uitgevoerd.

\subsection{Resultaten}

\begin{figure}[!ht]
  \centering
  \includegraphics[width=0.75\textwidth]{5_Experimenten_Resultaten/exp1_res}
  \caption{Resultaten van het experiment uitgevoerd in deel \ref{experiment:1}. De groene lijn staat voor de gemiddelde probabiliteit van een noot in de originele melodie, de rode lijn voor de gemiddelde probabiliteit van de theoritisch beste melodielijn in de toonaarden waarop getest werd en de blauwe lijn geeft de gemiddelde probabiliteit van een noot weer na uitvoer van het algoritme na een verschillend aantal iteraties.}
  \label{figuur:exp1}
\end{figure}

In figuur \ref{figuur:exp1} worden de restultaten weergegeven van dit experiment. De groene lijn op de figuur geeft de gemiddelde probabiliteit weer van alle melodie\"en waarop getest werd. De rode lijn geeft voor al deze melodie\"en de gemiddelde waarde mee voor het theoretisch beste muziekstuk dat volgens het RPK-model gemaakt kan worden in deze toonaard. Tot slot geeft de blauwe lijn de gemiddelde probabiliteit weer van een noot in de getransformeerde melodie, na toepassing van 1 tot 10 iteraties van de transformatie.\\
Er valt duidelijk op dat de eerste paar iteraties nog een redelijke verhoging van de probabiliteit teweeg brengt, maar dat na een vijftal iteraties gemiddeld gezien een soort van maximum bereikt is dat met deze transformatie kan bekomen worden. De reden dat deze waarde nog zo ver onder het theoretische maximum ligt, heeft er vooral mee te maken dat er maar 1 mogelijke transformatie is dat het algoritme mag gebruiken, hierdoor zijn er nog steeds maar een zeer beperkt aantal mogelijkheiden om een bepaalde noot te transformeren, en kunnen bijgevolg nog steeds de meeste noten niet bereikt worden vanuit eender welke noot.

\section{Transformaties combineren: meerdere transformaties, 1 iteratie}
\label{experiment:2}
\subsection{Beschrijving experiment}
Dit experiment heeft betrekking tot het algoritme dat besproken werd in onderdeel \ref{ETT:algo1}. Dit experiment gaat nagaan wat de invloed is van het aantal verschillende toegelaten transformaties op de consonantiescore van het totale muziekstuk. Zo zijn de vijf transformaties die gebruikt zullen worden weergegeven in tabel \ref{tabel:exp2}. Er wordt nu telkens slechts een iteratie van het algoritme uitgevoerd.\\
Het experiment wordt eerst uitgevoerd met slechts een mogelijke transformatie. Dit zal dan transformatie 1 uit de tabel zijn. Daarna wordt het experiment uitgevoerd met 2 mogelijke transformaties. Dit zullen dan de eerste twee transformaties uit de tabel zijn enz..

\begin{table}
  \centering
  \begin{tabular}{c | c c c c c c c c }
    Diff (mod 8) & 0 & 1 & 2 & 3 & 4 & 5 & 6 & 7 \\
    \hline
    \hline
    Verhoging transformatie 1 & 5 & -4 & 1 & -3 & 1 & 1 & 2 & 3 \\
    \hline
    Verhoging transformatie 2 & 1 & 3 & 4 & -5 & -1 & 6 & 5 & -1 \\
    \hline
    Verhoging transformatie 3 & 1 & 4 & 5 & -3 & 2 & -1 & 1 & 0 \\
    \hline
    Verhoging transformatie 2 & 4 & 6 & -2 & 4 & 2 & 6 & -4 & 2 \\
    \hline
    Verhoging transformatie -2 & -3 & -2 & 3 & -1 & 4 & -3 & 2 & -4 \\
  \end{tabular}
  \caption{Transformaties gebruikt in het experiment van onderdeel \ref{experiment:2}.}
  \label{tabel:exp2}
\end{table}

\subsection{Resultaten}

\begin{figure}[!ht]
  \centering
  \includegraphics[width=0.75\textwidth]{5_Experimenten_Resultaten/exp2_res}
  \caption{Resultaten van het experiment uitgevoerd in deel \ref{experiment:2}. De groene lijn staat voor de gemiddelde probabiliteit van een noot in de originele melodie, de rode lijn voor de gemiddelde probabiliteit van de theoritisch beste melodielijn in de toonaarden waarop getest werd en de blauwe lijn geeft de gemiddelde probabiliteit van een noot weer na uitvoer van het algoritme afhankelijk van het aantal transformaties dat ter beschikking gesteld was aan het algoritme.}
  \label{figuur:exp2}
\end{figure}

In figuur \ref{figuur:exp2} worden de restultaten weergegeven van dit experiment. Het is duidelijk dat een  hoger aantal transformaties ook telkens een betere score teruggeeft. Als we de resultaten van dit experiment vergelijken met dat uit onderdeel \ref{experiment:1}, dan merken we op dat het aantal transformaties een grotere impact heeft op de probabiliteit dan het aantal iteraties. De grootste reden hiervoor is dat een extra transformaties er voor kan zorgen dat er voor elke een noot in het muziekstuk een extra mogelijke noot is waarnaar hij getransformeerd kan worden. Dit zorgt voor enorm veel extra mogelijkheden waardoor er hogere probabiliteiten kunnen bekomen worden dan in het geval waarbij het aantal iteraties verhoogd wordt in plaats van het aantal mogelijke transformaties.  


\section{Transformaties combineren: minimum transformatie lengte}
\label{experiment:3}
\subsection{Beschrijving experiment}

\subsection{Resultaten}

\section{Transformaties combineren: Gelijkheid algoritmen voor transformatie lengte 1}
\label{experiment:4}
\subsection{Beschrijving experiment}

\subsection{Resultaten}

\section{Vergelijking transformatie scores}
\label{experiment:5}
\subsection{Beschrijving experiment}

\subsection{Resultaten}

\section{Invloed rij van Fibonacci op transformaties}
\label{experiment:6}
\subsection{Beschrijving experiment}

\subsection{Resultaten}

\section{Test Performantie algoritmes}
\label{experiment:7}
\subsection{Beschrijving experiment}

\subsection{Resultaten}

%%% Local Variables: 
%%% mode: latex
%%% TeX-master: "masterproef"
%%% End: 
